%===============================================================================
% LaTeX sjabloon voor de bachelorproef toegepaste informatica aan HOGENT
% Meer info op https://github.com/HoGentTIN/latex-hogent-report
%===============================================================================

\documentclass[dutch,dit,thesis]{hogentreport}


% TODO:
% - If necessary, replace the option `dit`' with your own department!
%   Valid entries are dbo, dbt, dgz, dit, dlo, dog, dsa, soa
% - If you write your thesis in English (remark: only possible after getting
%   explicit approval!), remove the option "dutch," or replace with "english".

\usepackage{lipsum} % For blind text, can be removed after adding actual content
\usepackage[backend=biber,style=apa]{biblatex}

%% Pictures to include in the text can be put isn the graphics/ folder
\graphicspath{{graphics/}}

%% For source code highlighting, requires pygments to be installed
%% Compile with the -shell-escape flag!
\usepackage[section]{minted}
\usepackage{amsmath}
\usepackage{enumitem}


%% If you compile with the make_thesis.{bat,sh} script, use the following
%% import instead:
%% \usepackage[section,outputdir=../output]{minted}
\usemintedstyle{solarized-light}
\definecolor{bg}{RGB}{253,246,227} %% Set the background color of the codeframe

%% Change this line to edit the line numbering style:
\renewcommand{\theFancyVerbLine}{\ttfamily\scriptsize\arabic{FancyVerbLine}}

%% Macro definition to load external java source files with \javacode{filename}:
\newmintedfile[javacode]{java}{
    bgcolor=bg,
    fontfamily=tt,
    linenos=true,
    numberblanklines=true,
    numbersep=5pt,
    gobble=0,
    framesep=2mm,
    funcnamehighlighting=true,
    tabsize=4,
    obeytabs=false,
    breaklines=true,
    mathescape=false
    samepage=false,
    showspaces=false,
    showtabs =false,
    texcl=false,
}

% Other packages not already included can be imported here

%%---------- Document metadata -------------------------------------------------
% TODO: Replace this with your own information
\author{Alexandra Stalmans}
\supervisor{Mevr. C. De Leenheer}
\cosupervisor{Dhr. T. Sanglet}
\title[]%
{Machine Learning pipeline voor Facial Image Analysis in camera-gebaseerde gezondheidsmetingen: Schatten van leeftijd en geslacht voor het beoordelen van geestelijke gezondheid}
\academicyear{\advance\year by -1 \the\year--\advance\year by 1 \the\year}
\examperiod{1}
\degreesought{\IfLanguageName{dutch}{Professionele bachelor in de toegepaste informatica}{Bachelor of applied computer science}}
\partialthesis{false} %% To display 'in partial fulfilment'
%\institution{Internshipcompany BVBA.}

%% Add global exceptions to the hyphenation here
\hyphenation{back-slash}

% The bibliography (style and settings are  found in hogentthesis.cls)

%\addbibresource{bachproef.bib}
%\addbibresource{../voorstel/voorstel.bib} %% Bibliography research proposal
\defbibheading{bibempty}{}

%% Prevent empty pages for right-handed chapter starts in twoside mode
\renewcommand{\cleardoublepage}{\clearpage}

\renewcommand{\arraystretch}{1.2}

%% Content starts here.
\begin{document}
    
    %---------- Front matter -------------------------------------------------------
    
    \frontmatter
    
    \hypersetup{pageanchor=false} %% Disable page numbering references
    %% Render a Dutch outer title page if the main language is English
    \IfLanguageName{english}{%
        %% If necessary, information can be changed here
        \degreesought{Professionele Bachelor toegepaste informatica}%
        \begin{otherlanguage}{dutch}%
            \maketitle%
        \end{otherlanguage}%
    }{}
    
    %% Generates title page content
    \maketitle
    \hypersetup{pageanchor=true}
    
    %%=============================================================================
%% Voorwoord
%%=============================================================================

\chapter*{\IfLanguageName{dutch}{Woord vooraf}{Preface}}%
\label{ch:voorwoord}

%% TODO:
%% Het voorwoord is het enige deel van de bachelorproef waar je vanuit je
%% eigen standpunt (``ik-vorm'') mag schrijven. Je kan hier bv. motiveren
%% waarom jij het onderwerp wil bespreken.
%% Vergeet ook niet te bedanken wie je geholpen/gesteund/... heeft

\lipsum[1-2]
    %%=============================================================================
%% Samenvatting
%%=============================================================================

% TODO: De "abstract" of samenvatting is een kernachtige (~ 1 blz. voor een
% thesis) synthese van het document.
%
% Een goede abstract biedt een kernachtig antwoord op volgende vragen:
%
% 1. Waarover gaat de bachelorproef?
% 2. Waarom heb je er over geschreven?
% 3. Hoe heb je het onderzoek uitgevoerd?
% 4. Wat waren de resultaten? Wat blijkt uit je onderzoek?
% 5. Wat betekenen je resultaten? Wat is de relevantie voor het werkveld?
%
% Daarom bestaat een abstract uit volgende componenten:
%
% - inleiding + kaderen thema
% - probleemstelling
% - (centrale) onderzoeksvraag
% - onderzoeksdoelstelling
% - methodologie
% - resultaten (beperk tot de belangrijkste, relevant voor de onderzoeksvraag)
% - conclusies, aanbevelingen, beperkingen
%
% LET OP! Een samenvatting is GEEN voorwoord!

%%---------- Nederlandse samenvatting -----------------------------------------
%
% TODO: Als je je bachelorproef in het Engels schrijft, moet je eerst een
% Nederlandse samenvatting invoegen. Haal daarvoor onderstaande code uit
% commentaar.
% Wie zijn bachelorproef in het Nederlands schrijft, kan dit negeren, de inhoud
% wordt niet in het document ingevoegd.

\IfLanguageName{english}{%
\selectlanguage{dutch}
\chapter*{Samenvatting}
\lipsum[1-4]
\selectlanguage{english}
}{}

%%---------- Samenvatting -----------------------------------------------------
% De samenvatting in de hoofdtaal van het document

\chapter*{\IfLanguageName{dutch}{Samenvatting}{Abstract}}

\lipsum[1-4]

    
    %---------- Inhoud, lijst figuren, ... -----------------------------------------
    
    \tableofcontents
    
    % In a list of figures, the complete caption will be included. To prevent this,
    % ALWAYS add a short description in the caption!
    %
    %  \caption[short description]{elaborate description}
    %
    % If you do, only the short description will be used in the list of figures
    
    \listoffigures
    
    % If you included tables and/or source code listings, uncomment the appropriate
    % lines.
    %\listoftables
    %\listoflistings
    
    % Als je een lijst van ingeingen of termen wil toevoegen, dan hoort die
    % hier thuis. Gebruik bijvoorbeeld de ``glossaries'' package.
    % https://www.overleaf.com/learn/latex/Glossaries
    
    
    %---------- Kern ---------------------------------------------------------------
    
    \mainmatter{}
    
    % De eerste hoofdstukken van een bachelorproef zijn meestal een inleiding op
    % het onderwerp, literatuurstudie en verantwoording methodologie.
    % Aarzel niet om een meer beschrijvende titel aan deze hoofdstukken te geven of
    % om bijvoorbeeld de inleiding en/of stand van zaken over meerdere hoofdstukken
    % te verspreiden!

    
    
    %%=============================================================================
%% Inleiding
%%=============================================================================

\chapter{\IfLanguageName{dutch}{Inleiding}{Introduction}}%
\label{ch:inleiding}

De onderzoeksvraag werd aangeboden door het bedrijf IntelliProve. IntelliProve biedt online gezondheidsoplossingen, een software die in staat is om binnen enkele seconden nauwkeurig gezondheidsparameters te bepalen, gebaseerd op een optische meting van het gezicht. Het doel van de bachelorproef is het ontwikkelen en implementeren van een robuust systeem voor het schatten van de leeftijd en het geslacht van personen op basis van gezichtsfoto’s, met behulp van machine learning-technieken. Dit project is van bijzonder belang voor het verbeteren van de beoordeling van de geestelijke gezondheidszorg door middel van camera-gebaseerde gezondheidsmetingen. Het onderzoek beoogt bij te dragen aan de vooruitgang op dit gebied door gebruik te maken van geavanceerde algoritmen om leeftijd en geslacht nauwkeurig te voorspellen aan de hand van gezichtsbeelden. De literatuurstudie biedt een inzicht in facial analysis, de bestaande machine learning modellen en hun functionaliteiten. De proof-of-concept zal bestaan uit het ontwikkelen van een machine learning pipeline die in staat is om leeftijd en geslacht te voorspellen op basis van bestaande datasets. De pipeline omvat verschillende image preprocessing technieken om de dataset voor te bereiden op de modeltraining. Om betrouwbaarheid en accuracy te garanderen, worden de modellen verfijnd en geoptimaliseerd om de hoogst mogelijke nauwkeurigheid te bereiken bij het schatten van leeftijd en geslacht.


\section{\IfLanguageName{dutch}{Probleemstelling}{Problem Statement}}%
\label{sec:probleemstelling}

Het onderzoek wordt uitgevoerd voor het bedrijf IntelliProve. IntelliProve biedt online gezondheidsoplossingen, een software die in staat is om binnen enkele seconden nauwkeurig gezondheidsparameters te bepalen, gebaseerd op een optische meting van het gezicht. De bachelorproef vormt een opstap naar een applicatie die in de toekomst uitgewerkt zal worden. 

\section{\IfLanguageName{dutch}{Onderzoeksvraag}{Research question}}%
\label{sec:onderzoeksvraag}
Het onderzoek beschrijft de ontwikkeling van een machine learning pipeline voor het schatten van leeftijd en geslacht. Specifiek voor het analyseren van gezichtsafbeeldingen in camera-gebaseerde gezondheidsmetingen. Voor het onderzoek werd volgende onderzoeksvraag opgesteld: 
\begin{itemize}
    \item Hoe kan een efficiënte machine learning-pipeline worden ontwikkeld en geoptimaliseerd voor het analyseren van gezichtsafbeeldingen in camera-gebaseerde gezondheidsmetingen, met als specifieke doelen het schatten van leeftijd en geslacht, toegepast om geestelijke gezondheid te beoordelen?
\end{itemize} \\
\\
De onderzoeksvraag kan opgedeeld worden in enkele deelvragen:

\begin{enumerate}
    \item Wat is de geschikte dataset voor het opgegeven probleem om zo veel mogelijk bias te vermijden? 
    \item Welke uitdagingen bestaan er al uit voorgaand onderzoek en moet rekening mee gehouden worden in het onderzoek?
    \item Uit welke stappen bestaat de machine learning pipeline?
     \begin{enumerate}
         \item Welke feature extractie en/of feature dimensionaliteitsreductie toepassingen zijn het meest geschikt voor het voorspellen van leeftijd en/of geslacht?
     \end{enumerate}
    \item Welke machine learning modellen zijn er mogelijk voor het voorspellen van leeftijd en geslacht? 
     \begin{enumerate}
        \item Wordt er 1 model gemaakt om leeftijd en geslacht te voorspellen of worden er 2 modellen gebruikt die zich elk richten tot een specifieke taak?
        \item Welk model, uit een vergelijkende studie, geeft de beste resultaten?
    \end{enumerate}
    \item Hoe kunnen we de performantie van een model meten?
    
\end{enumerate}



\section{\IfLanguageName{dutch}{Onderzoeksdoelstelling}{Research objective}}%
\label{sec:onderzoeksdoelstelling}

Het resultaat van de bachelorproef is een proof-of-concept die zal bestaan uit het ontwikkelen van een machine learning pipeline die in staat is om leeftijd en geslacht te voorspellen op basis van bestaande datasets. De pipeline omvat verschillende image preprocessing technieken om de dataset voor te bereiden op de modeltraining. Om betrouwbaarheid en accuracy te garanderen, worden de modellen verfijnd en geoptimaliseerd om de hoogst mogelijke nauwkeurigheid te bereiken bij het schatten van leeftijd en geslacht. Er wordt naar een zo hoog mogelijk nauwkeurigheid gestreefd, waarbij de conclusies uit dit onderzoek het belangrijkste zijn. Er kan aangegeven worden waarom bepaalde modellen goed werken of juist niet en of er in de toekomst nog verder onderzoek vereist is. 


\section{\IfLanguageName{dutch}{Opzet van deze bachelorproef}{Structure of this bachelor thesis}}%
\label{sec:opzet-bachelorproef}

% Het is gebruikelijk aan het einde van de inleiding een overzicht te
% geven van de opbouw van de rest van de tekst. Deze sectie bevat al een aanzet
% die je kan aanvullen/aanpassen in functie van je eigen tekst.

De rest van deze bachelorproef is als volgt opgebouwd:

In Hoofdstuk~\ref{ch:standvanzaken} wordt een overzicht gegeven van de stand van zaken binnen het onderzoeksdomein, op basis van een literatuurstudie.

In Hoofdstuk~\ref{ch:methodologie} wordt de methodologie toegelicht en worden de gebruikte onderzoekstechnieken besproken om een antwoord te kunnen formuleren op de onderzoeksvragen.

% TODO: Vul hier aan voor je eigen hoofstukken, één of twee zinnen per hoofdstuk

In Hoofdstuk~\ref{ch:conclusie}, tenslotte, wordt de conclusie gegeven en een antwoord geformuleerd op de onderzoeksvragen. Daarbij wordt ook een aanzet gegeven voor toekomstig onderzoek binnen dit domein.
    \chapter{\IfLanguageName{dutch}{Stand van zaken}{State of the art}}%
\label{ch:stand-van-zaken}

% Tip: Begin elk hoofdstuk met een paragraaf inleiding die beschrijft hoe
% dit hoofdstuk past binnen het geheel van de bachelorproef. Geef in het
% bijzonder aan wat de link is met het vorige en volgende hoofdstuk.

% Pas na deze inleidende paragraaf komt de eerste sectiehoofding.

Dit hoofdstuk bevat je literatuurstudie. De inhoud gaat verder op de inleiding, maar zal het onderwerp van de bachelorproef *diepgaand* uitspitten. De bedoeling is dat de lezer na lezing van dit hoofdstuk helemaal op de hoogte is van de huidige stand van zaken (state-of-the-art) in het onderzoeksdomein. Iemand die niet vertrouwd is met het onderwerp, weet nu voldoende om de rest van het verhaal te kunnen volgen, zonder dat die er nog andere informatie moet over opzoeken %\autocite{Pollefliet2011}.

Je verwijst bij elke bewering die je doet, vakterm die je introduceert, enz.\ naar je bronnen. In \LaTeX{} kan dat met het commando \texttt{$\backslash${textcite\{\}}} of \texttt{$\backslash${autocite\{\}}}. Als argument van het commando geef je de ``sleutel'' van een ``record'' in een bibliografische databank in het Bib\LaTeX{}-formaat (een tekstbestand). Als je expliciet naar de auteur verwijst in de zin (narratieve referentie), gebruik je \texttt{$\backslash${}textcite\{\}}. Soms is de auteursnaam niet expliciet een onderdeel van de zin, dan gebruik je \texttt{$\backslash${}autocite\{\}} (referentie tussen haakjes). Dit gebruik je bv.~bij een citaat, of om in het bijschrift van een overgenomen afbeelding, broncode, tabel, enz. te verwijzen naar de bron. In de volgende paragraaf een voorbeeld van elk.

%\textcite{Knuth1998} 
schreef een van de standaardwerken over sorteer- en zoekalgoritmen. Experten zijn het erover eens dat cloud computing een interessante opportuniteit vormen, zowel voor gebruikers als voor dienstverleners op vlak van informatietechnologie %~\autocite{Creeger2009}.

Let er ook op: het \texttt{cite}-commando voor de punt, dus binnen de zin. Je verwijst meteen naar een bron in de eerste zin die erop gebaseerd is, dus niet pas op het einde van een paragraaf.

\lipsum[7-20]

    %%=============================================================================
%% Methodologie
%%=============================================================================

\chapter{\IfLanguageName{dutch}{Methodologie}{Methodology}}%
\label{ch:methodologie}

%% TODO: In dit hoofstuk geef je een korte toelichting over hoe je te werk bent
%% gegaan. Verdeel je onderzoek in grote fasen, en licht in elke fase toe wat
%% de doelstelling was, welke deliverables daar uit gekomen zijn, en welke
%% onderzoeksmethoden je daarbij toegepast hebt. Verantwoord waarom je
%% op deze manier te werk gegaan bent.
%% 
%% Voorbeelden van zulke fasen zijn: literatuurstudie, opstellen van een
%% requirements-analyse, opstellen long-list (bij vergelijkende studie),
%% selectie van geschikte tools (bij vergelijkende studie, "short-list"),
%% opzetten testopstelling/PoC, uitvoeren testen en verzamelen
%% van resultaten, analyse van resultaten, ...
%%
%% !!!!! LET OP !!!!!
%%
%% Het is uitdrukkelijk NIET de bedoeling dat je het grootste deel van de corpus
%% van je bachelorproef in dit hoofstuk verwerkt! Dit hoofdstuk is eerder een
%% kort overzicht van je plan van aanpak.
%%
%% Maak voor elke fase (behalve het literatuuronderzoek) een NIEUW HOOFDSTUK aan
%% en geef het een gepaste titel.

\lipsum[21-25]


    
    % Voeg hier je eigen hoofdstukken toe die de ``corpus'' van je bachelorproef
    % vormen. De structuur en titels hangen af van je eigen onderzoek. Je kan bv.
    % elke fase in je onderzoek in een apart hoofdstuk bespreken.
    
    %\input{...}
    %...
    
    %%=============================================================================
%% Conclusie
%%=============================================================================

\chapter{Conclusie}%
\label{ch:conclusie}

% TODO: Trek een duidelijke conclusie, in de vorm van een antwoord op de
% onderzoeksvra(a)g(en). Wat was jouw bijdrage aan het onderzoeksdomein en
% hoe biedt dit meerwaarde aan het vakgebied/doelgroep? 
% Reflecteer kritisch over het resultaat. In Engelse teksten wordt deze sectie
% ``Discussion'' genoemd. Had je deze uitkomst verwacht? Zijn er zaken die nog
% niet duidelijk zijn?
% Heeft het onderzoek geleid tot nieuwe vragen die uitnodigen tot verder 
%onderzoek?

\lipsum[76-80]


    
    %---------- Bijlagen -----------------------------------------------------------
    
    \appendix
    
    \chapter{Onderzoeksvoorstel}
    
    Het onderwerp van deze bachelorproef is gebaseerd op een onderzoeksvoorstel dat vooraf werd beoordeeld door de promotor. Dat voorstel is opgenomen in deze bijlage.
    
    %% TODO: 
    \section*{Samenvatting}
    
    % Kopieer en plak hier de samenvatting (abstract) van je onderzoeksvoorstel.
    Gezichtsanalyse heeft de laatste jaren veel aandacht gekregen vanwege de brede toepassingen op verschillende gebieden, zoals gezondheidszorg, beveiliging en marketing.  
    De focus van de bachelorproef ligt op het ontwikkelen en implementeren van een robuust systeem voor het schatten van leeftijd en geslacht op basis van gezichtsfoto’s, met behulp van machine learning technieken. Dit onderzoek is van bijzonder belang voor het verbeteren van de beoordeling van geestelijke gezondheidszorg door camera-gebaseerde gezondheidsmetingen. Deze inspanningen dragen bij aan de doelstellingen van IntelliProve, een platform dat online gezondheidsoplossingen biedt. Er wordt onderzocht welke machine learning technieken de beste resultaten tonen en uit welke elementen de volledige pipeline zal bestaan.   
    In de eerste fase van het onderzoek wordt een literatuurstudie uitgevoerd om uit bestaand onderzoek de gebruikte modellen te verkennen en hun sterktes, limitaties en performantie te achterhalen. In de proof-of-concept gebeurt eerst de preprocessing, feature extractie en gezichtsdetectie. Daarna worden de modellen uit de literatuurstudie getest op bestaande datasets en wordt een prestatie-evaluatie opgesteld op basis van metrieken zoals accuracy, precision, recall, F1-score en ROC curve. Het verwachte resultaat is een robuuste machine learning pipeline waarmee IntelliProve de leeftijd en het geslacht kan voorspellen op basis van een gezichtsafbeelding. 
    
    % Verwijzing naar het bestand met de inhoud van het onderzoeksvoorstel
    %---------- Inleiding ---------------------------------------------------------

\section{Introductie}%
\label{sec:introductie}

De onderzoeksvraag werd aangeboden door het bedrijf IntelliProve. IntelliProve biedt online gezondheidsoplossingen, een software die in staat is om binnen enkele seconden nauwkeurig gezondheidsparameters te bepalen, gebaseerd op een optische meting van het gezicht.  
Het doel van de bachelorproef is het ontwikkelen en implementeren van een robuust systeem voor het schatten van de leeftijd en het geslacht van personen op basis van gezichtsfoto's, met behulp van machine learning-technieken. 
Dit project is van bijzonder belang voor het verbeteren van de beoordeling van de geestelijke gezondheidszorg door middel van camera-gebaseerde gezondheidsmetingen. Het onderzoek beoogt bij te dragen aan de vooruitgang op dit gebied door gebruik te maken van geavanceerde algoritmen om leeftijd en geslacht nauwkeurig te voorspellen aan de hand van gezichtsbeelden.  
De literatuurstudie biedt een inzicht in facial analysis, de bestaande machine learning modellen en hun functionaliteiten. De proof-of-concept zal bestaan uit het ontwikkelen van een machine learning pipeline dat in staat is om leeftijd en geslacht te voorspellen op basis van bestaande datasets. De pipeline omvat verschillende image preprocessing technieken om de dataset voor te bereiden op de modeltraining. Om betrouwbaarheid en accuracy te garanderen, worden de modellen verfijnd en geoptimaliseerd om de hoogst mogelijke nauwkeurigheid te bereiken bij het schatten van leeftijd en geslacht.

\section{State-of-the-art}%
\label{sec:state-of-the-art}

Hier beschrijf je de \emph{state-of-the-art} rondom je gekozen onderzoeksdomein, d.w.z.\ een inleidende, doorlopende tekst over het onderzoeksdomein van je bachelorproef. Je steunt daarbij heel sterk op de professionele \emph{vakliteratuur}, en niet zozeer op populariserende teksten voor een breed publiek. Wat is de huidige stand van zaken in dit domein, en wat zijn nog eventuele open vragen (die misschien de aanleiding waren tot je onderzoeksvraag!)?

Je mag de titel van deze sectie ook aanpassen (literatuurstudie, stand van zaken, enz.). Zijn er al gelijkaardige onderzoeken gevoerd? Wat concluderen ze? Wat is het verschil met jouw onderzoek?

Verwijs bij elke introductie van een term of bewering over het domein naar de vakliteratuur, bijvoorbeeld~\autocite{Hykes2013}! Denk zeker goed na welke werken je refereert en waarom.

Draag zorg voor correcte literatuurverwijzingen! Een bronvermelding hoort thuis \emph{binnen} de zin waar je je op die bron baseert, dus niet er buiten! Maak meteen een verwijzing als je gebruik maakt van een bron. Doe dit dus \emph{niet} aan het einde van een lange paragraaf. Baseer nooit teveel aansluitende tekst op eenzelfde bron.

Als je informatie over bronnen verzamelt in JabRef, zorg er dan voor dat alle nodige info aanwezig is om de bron terug te vinden (zoals uitvoerig besproken in de lessen Research Methods).

% Voor literatuurverwijzingen zijn er twee belangrijke commando's:
% \autocite{KEY} => (Auteur, jaartal) Gebruik dit als de naam van de auteur
%   geen onderdeel is van de zin.
% \textcite{KEY} => Auteur (jaartal)  Gebruik dit als de auteursnaam wel een
%   functie heeft in de zin (bv. ``Uit onderzoek door Doll & Hill (1954) bleek
%   ...'')

Je mag deze sectie nog verder onderverdelen in subsecties als dit de structuur van de tekst kan verduidelijken.

%---------- Methodologie ------------------------------------------------------
\section{Methodologie}%
\label{sec:methodologie}

\subsection{Requirements}
\label{sub:requirements}
In de eerste week wordt nagevraagd aan belanghebbenden van IntelliProve aan welke criteria de modellen moeten voldoen. Alle data (gezichtsfoto's) worden verzameld. Er wordt onder andere nagegaan over welke functionaliteiten de modellen moeten beschikken en wat de verwachte prestatievereisten zijn. 
Als resultaat verwerven we een lijst van alle functionele en niet-functionele requirements, geordend volgens belang. 

\subsection{Literatuurstudie}
\label{sub:literatuurstudie}
De literatuurstudie omvat een diepgaande verkenning van facial analysis technieken en machine learning modellen. 
Deze fase biedt inzicht in de verschillende methoden voor het extraheren van gezichtskenmerken en image preprocessing technieken, specifiek met betrekking tot het schatten van leeftijd en geslacht.
Het doel is om kennis uit bestaand onderzoek te vergaren om effectieve methodologieën te identificeren in de huidige benaderingen van facial analysis. 
Het eindresultaat van deze fase, die 3 weken duurt, is een samenvatting van de belangrijkste bevindingen uit de literatuurstudie, die als basis zal dienen voor de proof-of-concept.
\subsection{Proof-of-concept}
\label{sub:proof-of-concept}
Deze fase start met het verzamelen en analyseren van de datasets. Technieken zoals normalisatie, scaling, feature-extractie en data augmentation worden gebruikt om de dataset voor te bereiden op modeltraining.
Vervolgens worden verschillende machine learning algoritmen geselecteerd op basis van de bevindingen uit de literatuurstudie. De modellen worden getraind en geëvalueerd op basis van de vooropgestelde requirements. 
De pipeline wordt beoordeeld op betrouwbaarheid en precisie op basis van de opgestelde requirements. Metrieken zoals accuracy, precision, recall, F1-score en ROC-curves worden gebruikt om de modellen te evalueren.
Er worden cross-validatie technieken gebruikt om de robuustheid van de modellen en de generalisatie van nieuwe data te testen. 
Deze fase vereist dan ook veel tijd en zal 6 weken duren. Het resultaat van deze fase is een proof-of-concept die bestaat uit een machine learning pipeline, die stappen voor het preprocessen van gegevens en getrainde modellen integreert voor het schatten van leeftijd en geslacht op basis van gezichtsfoto's.

\subsection{conclusie}
\label{sub:conclusie}
In de conclusiefase worden de resultaten van de evaluatie grondig geanalyseerd. De prestaties van de ontwikkelde modellen worden beoordeeld, waarbij hun sterke punten en beperkingen worden benadrukt. Het belang van een nauwkeurige schatting van leeftijd en geslacht bij de beoordeling van de geestelijke gezondheid en de implicaties voor de gezondheidszorg worden besproken. Er worden aanbevelingen gedaan voor mogelijke verbeteringen of toekomstige onderzoeksrichtingen op basis van de bevindingen en beperkingen van het project.
\subsection{Afwerken scriptie}
\label{sub:afwerken_scriptie}
De laatste fase, die 2 weken duurt, omvat het afwerken van de bachelorproef. Dit is het eindresultaat van het geleverde onderzoek met een proof-of-concept die zal worden ingediend. 


%---------- Verwachte resultaten ----------------------------------------------
\section{Verwacht resultaat, conclusie}%
\label{sec:verwachte_resultaten}

Hier beschrijf je welke resultaten je verwacht. Als je metingen en simulaties uitvoert, kan je hier al mock-ups maken van de grafieken samen met de verwachte conclusies. Benoem zeker al je assen en de onderdelen van de grafiek die je gaat gebruiken. Dit zorgt ervoor dat je concreet weet welk soort data je moet verzamelen en hoe je die moet meten.

Wat heeft de doelgroep van je onderzoek aan het resultaat? Op welke manier zorgt jouw bachelorproef voor een meerwaarde?

Hier beschrijf je wat je verwacht uit je onderzoek, met de motivatie waarom. Het is \textbf{niet} erg indien uit je onderzoek andere resultaten en conclusies vloeien dan dat je hier beschrijft: het is dan juist interessant om te onderzoeken waarom jouw hypothesen niet overeenkomen met de resultaten.


    
    %%---------- Andere bijlagen --------------------------------------------------
    % TODO: Voeg hier eventuele andere bijlagen toe. Bv. als je deze BP voor de
    % tweede keer indient, een overzicht van de verbeteringen t.o.v. het origineel.
    %\input{...}
    
    %%---------- Backmatter, referentielijst ---------------------------------------
    
    
    \backmatter{}
    
    \setlength\bibitemsep{2pt} %% Add Some space between the bibliograpy entries
    \printbibliography[heading=bibintoc]
    
\end{document}
