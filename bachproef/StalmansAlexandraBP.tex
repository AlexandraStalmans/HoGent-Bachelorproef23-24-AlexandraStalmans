%===============================================================================
% LaTeX sjabloon voor de bachelorproef toegepaste informatica aan HOGENT
% Meer info op https://github.com/HoGentTIN/latex-hogent-report
%===============================================================================

\documentclass[dutch,dit,thesis]{hogentreport}


% TODO:
% - If necessary, replace the option `dit`' with your own department!
%   Valid entries are dbo, dbt, dgz, dit, dlo, dog, dsa, soa
% - If you write your thesis in English (remark: only possible after getting
%   explicit approval!), remove the option "dutch," or replace with "english".

\usepackage{lipsum} % For blind text, can be removed after adding actual content
\usepackage[backend=biber,style=apa]{biblatex}

%% Pictures to include in the text can be put isn the graphics/ folder
\graphicspath{{graphics/}}

%% For source code highlighting, requires pygments to be installed
%% Compile with the -shell-escape flag!
\usepackage[section]{minted}
\usepackage{amsmath}
\usepackage{enumitem}
\usepackage{listings}
\usepackage{xcolor}
\definecolor{backcolour}{rgb}{0.95,0.95,0.92}
\definecolor{hogent-darkgreen}{RGB}{22,176,165}
\definecolor{hogent-pink}{RGB}{241,157,160}
\definecolor{hogent-ochre}{RGB}{250,188,50}
\definecolor{hogent-orange}{RGB}{239,135,103}
\definecolor{hogent-purple}{RGB}{187,144,189}
\definecolor{hogent-blue}{RGB}{76,162,213}
\definecolor{hogent-lightgreen}{RGB}{165,202,114}
\definecolor{hogent-brown}{RGB}{216,176,131}
\definecolor{hogent-grey}{RGB}{195,187,175}
\definecolor{hogent-yellow}{RGB}{244,222,0}

\lstdefinestyle{mystyle}{
    backgroundcolor=\color{backcolour},   
    commentstyle=\color{hogent-darkgreen},
    keywordstyle=\color{hogent-purple},
    numberstyle=\tiny\color{hogent-grey},
    stringstyle=\color{codepurple},
    caption={\color{hogent-blue}{\protect\caption}},
    basicstyle=\ttfamily\footnotesize,
    breakatwhitespace=false,         
    breaklines=true,                 
    captionpos=b,                    
    keepspaces=true,                 
    numbers=left,                    
    numbersep=5pt,                  
    showspaces=false,                
    showstringspaces=false,
    showtabs=false,                  
    tabsize=2
}

\lstset{style=mystyle}


%% If you compile with the make_thesis.{bat,sh} script, use the following
%% import instead:
%% \usepackage[section,outputdir=../output]{minted}
\usemintedstyle{solarized-light}
\definecolor{bg}{RGB}{253,246,227} %% Set the background color of the codeframe

%% Change this line to edit the line numbering style:
\renewcommand{\theFancyVerbLine}{\ttfamily\scriptsize\arabic{FancyVerbLine}}

%% Macro definition to load external java source files with \javacode{filename}:
\newmintedfile[javacode]{java}{
    bgcolor=bg,
    fontfamily=tt,
    linenos=true,
    numberblanklines=true,
    numbersep=5pt,
    gobble=0,
    framesep=2mm,
    funcnamehighlighting=true,
    tabsize=4,
    obeytabs=false,
    breaklines=true,
    mathescape=false
    samepage=false,
    showspaces=false,
    showtabs =false,
    texcl=false,
}

% Other packages not already included can be imported here

%%---------- Document metadata -------------------------------------------------
% TODO: Replace this with your own information
\author{Alexandra Stalmans}
\supervisor{Mevr. C. De Leenheer}
\cosupervisor{Dhr. T. Sanglet}
\title[]%
{Machine Learning pipeline voor Facial Image Analysis in camera-gebaseerde gezondheidsmetingen: Schatten van leeftijd en geslacht voor het beoordelen van geestelijke gezondheid}
\academicyear{\advance\year by -1 \the\year--\advance\year by 1 \the\year}
\examperiod{1}
\degreesought{\IfLanguageName{dutch}{Professionele bachelor in de toegepaste informatica}{Bachelor of applied computer science}}
\partialthesis{false} %% To display 'in partial fulfilment'
%\institution{Internshipcompany BVBA.}

%% Add global exceptions to the hyphenation here
\hyphenation{back-slash}

% The bibliography (style and settings are  found in hogentthesis.cls)

\addbibresource{../bachproef/bachproef.bib}
\addbibresource{../voorstel/voorstel.bib} %% Bibliography research proposal
\defbibheading{bibempty}{}

%% Prevent empty pages for right-handed chapter starts in twoside mode
\renewcommand{\cleardoublepage}{\clearpage}

\renewcommand{\arraystretch}{1.2}

%% Content starts here.
\begin{document}
    
    %---------- Front matter -------------------------------------------------------
    
    \frontmatter
    
    \hypersetup{pageanchor=false} %% Disable page numbering references
    %% Render a Dutch outer title page if the main language is English
    \IfLanguageName{english}{%
        %% If necessary, information can be changed here
        \degreesought{Professionele Bachelor toegepaste informatica}%
        \begin{otherlanguage}{dutch}%
            \maketitle%
        \end{otherlanguage}%
    }{}
    
    %% Generates title page content
    \maketitle
    \hypersetup{pageanchor=true}
    
    \input{voorwoord}
    \input{samenvatting}
    
    %---------- Inhoud, lijst figuren, ... -----------------------------------------
    
    \tableofcontents
    
    % In a list of figures, the complete caption will be included. To prevent this,
    % ALWAYS add a short description in the caption!
    %
    %  \caption[short description]{elaborate description}
    %
    % If you do, only the short description will be used in the list of figures
    
    \listoffigures
    
    % If you included tables and/or source code listings, uncomment the appropriate
    % lines.
    %\listoftables
    \listoflistings
    
    % Als je een lijst van ingeingen of termen wil toevoegen, dan hoort die
    % hier thuis. Gebruik bijvoorbeeld de ``glossaries'' package.
    % https://www.overleaf.com/learn/latex/Glossaries
    
    
    %---------- Kern ---------------------------------------------------------------
    
    \mainmatter{}
    
    % De eerste hoofdstukken van een bachelorproef zijn meestal een inleiding op
    % het onderwerp, literatuurstudie en verantwoording methodologie.
    % Aarzel niet om een meer beschrijvende titel aan deze hoofdstukken te geven of
    % om bijvoorbeeld de inleiding en/of stand van zaken over meerdere hoofdstukken
    % te verspreiden!

    
    
    %%=============================================================================
%% Inleiding
%%=============================================================================

\chapter{\IfLanguageName{dutch}{Inleiding}{Introduction}}%
\label{ch:inleiding}

De onderzoeksvraag werd aangeboden door het bedrijf IntelliProve. IntelliProve biedt online gezondheidsoplossingen, een software die in staat is om binnen enkele seconden nauwkeurig gezondheidsparameters te bepalen, gebaseerd op een optische meting van het gezicht. Het doel van de bachelorproef is het ontwikkelen en implementeren van een robuust systeem voor het schatten van de leeftijd en het geslacht van personen op basis van gezichtsfoto’s, met behulp van machine learning-technieken. Dit project is van bijzonder belang voor het verbeteren van de beoordeling van de geestelijke gezondheidszorg door middel van camera-gebaseerde gezondheidsmetingen. Het onderzoek beoogt bij te dragen aan de vooruitgang op dit gebied door gebruik te maken van geavanceerde algoritmen om leeftijd en geslacht nauwkeurig te voorspellen aan de hand van gezichtsbeelden. De literatuurstudie biedt een inzicht in facial analysis, de bestaande machine learning modellen en hun functionaliteiten. De proof-of-concept zal bestaan uit het ontwikkelen van een machine learning pipeline die in staat is om leeftijd en geslacht te voorspellen op basis van bestaande datasets. De pipeline omvat verschillende image preprocessing technieken om de dataset voor te bereiden op de modeltraining. Om betrouwbaarheid en accuracy te garanderen, worden de modellen verfijnd en geoptimaliseerd om de hoogst mogelijke nauwkeurigheid te bereiken bij het schatten van leeftijd en geslacht.


\section{\IfLanguageName{dutch}{Probleemstelling}{Problem Statement}}%
\label{sec:probleemstelling}

Het onderzoek wordt uitgevoerd voor het bedrijf IntelliProve. IntelliProve biedt online gezondheidsoplossingen, een software die in staat is om binnen enkele seconden nauwkeurig gezondheidsparameters te bepalen, gebaseerd op een optische meting van het gezicht. De bachelorproef vormt een opstap naar een applicatie die in de toekomst uitgewerkt zal worden. 

\section{\IfLanguageName{dutch}{Onderzoeksvraag}{Research question}}%
\label{sec:onderzoeksvraag}
Het onderzoek beschrijft de ontwikkeling van een machine learning pipeline voor het schatten van leeftijd en geslacht. Specifiek voor het analyseren van gezichtsafbeeldingen in camera-gebaseerde gezondheidsmetingen. Voor het onderzoek werd volgende onderzoeksvraag opgesteld: 
\begin{itemize}
    \item Hoe kan een efficiënte machine learning-pipeline worden ontwikkeld en geoptimaliseerd voor het analyseren van gezichtsafbeeldingen in camera-gebaseerde gezondheidsmetingen, met als specifieke doelen het schatten van leeftijd en geslacht, toegepast om geestelijke gezondheid te beoordelen?
\end{itemize} \\
\\
De onderzoeksvraag kan opgedeeld worden in enkele deelvragen:

\begin{enumerate}
    \item Wat is de geschikte dataset voor het opgegeven probleem om zo veel mogelijk bias te vermijden? 
    \item Welke uitdagingen bestaan er al uit voorgaand onderzoek en moet rekening mee gehouden worden in het onderzoek?
    \item Uit welke stappen bestaat de machine learning pipeline?
     \begin{enumerate}
         \item Welke feature extractie en/of feature dimensionaliteitsreductie toepassingen zijn het meest geschikt voor het voorspellen van leeftijd en/of geslacht?
     \end{enumerate}
    \item Welke machine learning modellen zijn er mogelijk voor het voorspellen van leeftijd en geslacht? 
     \begin{enumerate}
        \item Wordt er 1 model gemaakt om leeftijd en geslacht te voorspellen of worden er 2 modellen gebruikt die zich elk richten tot een specifieke taak?
        \item Welk model, uit een vergelijkende studie, geeft de beste resultaten?
    \end{enumerate}
    \item Hoe kunnen we de performantie van een model meten?
    
\end{enumerate}



\section{\IfLanguageName{dutch}{Onderzoeksdoelstelling}{Research objective}}%
\label{sec:onderzoeksdoelstelling}

Het resultaat van de bachelorproef is een proof-of-concept die zal bestaan uit het ontwikkelen van een machine learning pipeline die in staat is om leeftijd en geslacht te voorspellen op basis van bestaande datasets. De pipeline omvat verschillende image preprocessing technieken om de dataset voor te bereiden op de modeltraining. Om betrouwbaarheid en accuracy te garanderen, worden de modellen verfijnd en geoptimaliseerd om de hoogst mogelijke nauwkeurigheid te bereiken bij het schatten van leeftijd en geslacht. Er wordt naar een zo hoog mogelijk nauwkeurigheid gestreefd, waarbij de conclusies uit dit onderzoek het belangrijkste zijn. Er kan aangegeven worden waarom bepaalde modellen goed werken of juist niet en of er in de toekomst nog verder onderzoek vereist is. 


\section{\IfLanguageName{dutch}{Opzet van deze bachelorproef}{Structure of this bachelor thesis}}%
\label{sec:opzet-bachelorproef}

% Het is gebruikelijk aan het einde van de inleiding een overzicht te
% geven van de opbouw van de rest van de tekst. Deze sectie bevat al een aanzet
% die je kan aanvullen/aanpassen in functie van je eigen tekst.

De rest van deze bachelorproef is als volgt opgebouwd:

In Hoofdstuk~\ref{ch:standvanzaken} wordt een overzicht gegeven van de stand van zaken binnen het onderzoeksdomein, op basis van een literatuurstudie.

In Hoofdstuk~\ref{ch:methodologie} wordt de methodologie toegelicht en worden de gebruikte onderzoekstechnieken besproken om een antwoord te kunnen formuleren op de onderzoeksvragen.

% TODO: Vul hier aan voor je eigen hoofstukken, één of twee zinnen per hoofdstuk
In Hoofdstuk~\ref{ch:proofofconcept} wordt de Proof of Concept uitgewerkt en de code toegelicht. 

In Hoofdstuk~\ref{ch:conclusie}, tenslotte, wordt de conclusie gegeven en een antwoord geformuleerd op de onderzoeksvragen. Daarbij wordt ook een aanzet gegeven voor toekomstig onderzoek binnen dit domein.
    \chapter{\IfLanguageName{dutch}{Stand van zaken}{State-of-the-art}}%
\label{ch:standvanzaken}

Gezichtsanalyse bestaat uit het definiëren van menselijke gezichten in real time, met behulp van computeralgoritmen en machine learning technieken. Voor mensen en computersystemen bevat een gezichtsbeeld details zoals leeftijd, geslacht, stemming, afkomst, et cetera. \\
\\ 
Gezichtsanalyse omvat dan ook het lokaliseren en meten van gezichtskenmerken in een afbeelding. Vervolgens wordt het gezichtsbeeld geanalyseerd voor het extraheren van kenmerken \autocite{Sanil2023}. Gezichtsanalyse speelt dan ook een grote rol in real world applicaties, zoals animatie, identiteitsverificatie, medische diagnose, et cetera. Ondanks het bestaande onderzoekswerk over dit onderwerp, is gezichtsanalyse nog steeds een lastige taak vanwege verschillende factoren zoals veranderingen in hoek, gezichtsuitdrukkingen en achtergrond \autocite{Siddiqi2022}. \\

Volgens onderzoek van \textcite{Basystiuk2023} is beeldherkenning een simpele procedure die uit slechts 3 stappen bestaat. 
\begin{enumerate}
    \item Preprocessing: We voegen filters toe aan de afbeelding om het geschikter te maken voor herkenning.
    \item Feature Extractie: We identificeren belangrijke data en behouden deze om mee verder te werken. Dit wordt verder besproken in \ref{sub:gezichtsdetectie}.
    \item Classificatie: het analyseren en identificeren van de data na de feature extractie.
\end{enumerate}


\section{Gezichtsdetectie}\label{sub:gezichtsdetectie}
Gezichtsdetectie is de eerste stap in het bepalen van de gezichtsfeatures. Deze features zijn de interessante delen van het gezicht , bijvoorbeeld ogen, neus en mond en worden ook wel eens landmarks genoemd \autocite{Coppens2018}. In dit onderzoek raadt de auteur 3 gezichtsdetectieservices aan: Microsoft Face API, Face++ en Kairos. Deze zijn alle 3 ook geschikt om geslacht en leeftijd te voorspellen. Er wordt met 3 services tegelijk gewerkt om elkaars beperkingen op te vangen.
In het onderzoek van \textcite{Sanil2023} berekenen ze de landmarks op 2 verschillende methoden: Euclidische afstand en Geodetische afstand. \\

\textcite{Sanil2023} stelt 468 landmarks visueel voor, in plaats van 68, om de accuracy te verbeteren. Dit door gebruik te maken van de Mediapipe library (\textcite{Zubair2021}). Op figuur {~\ref{fig:landmarks}} staan de belangrijkste antropometrische landmarks aangeduid . Het is noodzakelijk om de belangrijkste features te identificeren die kunnen leiden tot het correct voorspellen van de leeftijd of het geslacht. Om de leeftijd te voorspellen kunnen we bijvoorbeeld de rimpels op de gezichtsfoto analyseren \autocite{Kwon1994}. 
\begin{figure}
    \centering
    \includegraphics{graphics/faciallandmarks.png}
    \caption[Belangrijkste anthropometrische landmarks]{\label{fig:landmarks}De belangrijkste anthropometrische landmarks\autocite{Sanil2023}.}
\end{figure}

\section{Feature extractie} \label{sec:featextractie}
\subsection{Normalisatie}\label{sub:normalisatie}
In onderzoek van \autocite{Chen2011} voor het voorspellen van leeftijd op basis van afbeeldingen werden de gezichtsafbeeldingen genormaliseerd voor de feature extraction plaatsvond. De geometrie van de afbeeldingen kan worden genormaliseerd op basis van de gedetecteerde features, zoals oog of mond coördinatie. De auteur maakt gebruik van een mask om de pixels die zich niet in de ovaal van de typsiche gezichtsregio bevinden te verwijderen. Dit gaat dan bijvoorbeeld over haar en hemdskragen. Zo blijven enkel de belangrijkste features van het gezicht over. Figuur {~\ref{fig:beforenormalisation}} geeft een voorbeeld van een dataset met gezichtsfoto's. In figuur {~\ref{fig:afternormalisation}} worden deze afbeeldingen genormaliseerd, waardoor er minder variaties overblijven in de afbeeldingen en we bijvoorbeeld al geen achtergrond overhouden.
\begin{figure}
    \centering
    \includegraphics{graphics/beforenorm.PNG}
    \caption[Gezichtsafbeeldingen voor normalisatie]{\label{fig:beforenormalisation}Gezichtsafbeeldingen voor normalisatie\autocite{Chen2011}.}
\end{figure}
\begin{figure}
    \centering
    \includegraphics{graphics/afternorm.PNG}
    \caption[Gezichtsafbeeldingen na normalisatie]{\label{fig:afternormalisation}Gezichtsafbeeldingen na het toepassen van normalisatie\autocite{Chen2011}.}
\end{figure}  \\
De gezichtsafbeeldingen kunnen ook genormaliseerd worden op basis van de coördinaten van de ogen, zodat het centrum van beide ogen van de afbeeldingen op een vaste positie liggen \autocite{Chen2011}. 

\subsection{AAM}\label{sub:aam}
In het onderzoek van \textcite{Lakshmiprabha2016} worden de Active Appearance Model (AAM) features gebruikt in de feature extractie stap. Deze features zijn geschikt om het effect van veroudering weer te geven voor het voorspellen van geslacht. De features die we van AAM krijgen hebben zowel vorm- als textuurinformatie die geschikter is voor veroudering. Figuur {~\ref{fig:aam}} geeft een representatie van de FG-NET database na het toepassen van de AAM-features. Het resultaat is vergelijkbaar met de normalisatie van \textcite{Chen2011} op figuur {~\ref{fig:afternormalisation}}.
\begin{figure}
    \centering
    \includegraphics{graphics/AAM.PNG}
    \caption[AAM-features]{\label{fig:aam}Cohn Kanada database na toepassen van AAM \autocite{Lakshmiprabha2016}.}
\end{figure} 

\subsection{LBP} \label{sub:lbp}
Local Binary Patterns (LBP) is een textuurdescriptor die gebruikt kan worden om gezichten te representeren, aangezien een gezichtsfbeelding kan worden beschouwd als een samenstelling van microtextuurpatronen \autocite{SanchezLopez2010}. De LBP operator kent een label toe aan elke pixel van een afbeelding door een threshold te nemen van de 3x3 buren met de centrale piwelwaarde en het resultaat te beschouwen als een binair getal. Het resultaat na de toepassing van LBP in het onderzoek van \textcite{SanchezLopez2010} is weergegeven op figuur {~\ref{fig:lbp}}.
\begin{figure}
    \centering
    \includegraphics{graphics/LBP.PNG}
    \caption[AAM-features]{\label{fig:lbp}Toepassen van LBP op een 16x16 gezichtsafbeelding resulteert in 14x14 label afbeelding \autocite{SanchezLopez2010}.}
\end{figure}
\\
In het onderzoek van \textcite{Chen2011} wordt LBP toegepast om de gezichtsfeatures te extraheren uit elk blok die de leeftijdsinformatie bevat. Het resultaat hiervan is een feature dimensie vector van 59. Deze features worden dan samengevoegd in een holistische vector. 

\section{Feature dimensionaliteitsreductie} \label{sec:featuredimred}
De herkenning van gezicht focust zich voornamelijk op het detecteren van individuele features in het gezicht, zoals ogen, hoofdomtrek, mond of het definiëren van het model van gezicht door positie, grootte of relatie tussen de features \autocite{Lin2006}. Het extraheren van de features speelt een grote rol in de preprocessing fase. \\
\\
De Principal Component Analyse (PCA) wordt veelal gebruikt voor gezichtsherkenning. \gls{pca} is een unsupervised, statistische techniek die gebruikt wordt voor dimensionaliteitsreductie in machine learning \autocite{SalihHasan2021}. Het reduceert het hoge aantal dimensies in een grote dataset naar een lager aantal dimensies om de opslag en het verwerkingsproces te versnellen. Hierdoor is de data makkelijker te interpreteren en sneller te verwerken. De techniek behoudt het grootste aantal informatie en verwijdert redundante ruis en data. Het grote nadeel van PCA is dat de lineaire transformatie die wordt uitgevoerd om de grootste varianties te behouden, niet altijd goed geschikt is voor het op te lossen probleem, namelijk de classificatie van geslacht \autocite{Wang2010}. \\
De feature reductie vindt typisch plaats na de feature extractie fase en voor de eigenlijke classificatie \autocite{Lakshmiprabha2016}.

\section{Bestaande uitdagingen} \label{sec:uitdagingen}
Ondanks het uitgebreide onderzoekwerk, heersen er nog vele uitdagingen op het vlak van gezichtsanalyse. Deze uitdagingen zijn het gevolg van verschillende factoren zoals gezichtsuitdrukkingen, ruis, belichting, et cetera. Om de nauwkeurigheid van de gezichtsherkenning te verbeteren, is het belangrijk om taken van de gezichtsanalyse met elkaar te correleren. Het is bijvoorbeeld zeer waarschijnlijk dat mannen een baard of snor kunnen hebben, maar vrouwen en kinderen niet \autocite{Siddiqi2022}. 
\\
Ook is het detecteren van de locatie waar de gezichten zich bevinden op de afbeelding een uitdaging. Dit wordt vaak meegenomen in de preprocessing stap van het analysesysteem \autocite{Jiang2008}.
\\
De problemen situeren zich niet enkel op het vlak van de afbeelding, maar ook bij de machine learning modellen. Eén van de grootste problemen hierbij is overfitting. Dit gebeurt wanneer we het algoritme te veel trainen, waardoor het te hard lijkt op de trainingsdata. Hierdoor zal het systeem falen wanneer we nieuwe data willen classificeren \autocite{Coppens2018}.

\section{Bestaande Machine-Learning technieken} \label{sec:bestaandeml}
Supervised learning is de methode die input mapt naar een output, gebaseerd op een voorbeeld \autocite{Rustam2018}. Supervised learning is gericht op het voorspellen van de waarde van variabelen of labels, op basis van de input features. Het wordt gewoonlijk gebruikt voor classificatie, benadering, modelleren of identificatie. Het groeperen van mannen of vrouwen op basis van gezichtsherkenning is een classificatieprobleem. Er zijn 2 opties: man of vrouw, ofwel 0 of 1. Dit maakt supervised learning de geschikte methode voor de voorspelling van geslacht. Bij supervised learning bestaat de trainingsdata uit een reeks trainingsvoorbeelden, waarbij elk voorbeeld bestaat uit een input en een verwachte output waarde \autocite{Shah2012}. Het algoritme analyseert de trainingsdata en voorspelt dan de correcte output categorie voor de gegeven data input.  

\subsection{Extreme Gradient Boost}
\label{sub:xgboost}
Extreme Gradient Boost, ofwel XGBoost, is een supervised learning model, waar trainingdata, met meerdere features, x\textsubscript{i} wordt gebruikt om target waarde y\textsubscript{i} te voorspellen \autocite{XGBoost2023}. Uit onderzoek van \autocite{Chen2023}, waarbij vermoeidheid werd voorspeld op basis van gezichtsafbeeldingen met XGBoost, wordt een boom aangemaakt op basis van de geleverde features. Dit model gaf een hoge accuracy en is minder gevoelig voor noise. Uit onderzoek naar gezichtsdetectie van \textcite{Sanil2023} bleek dat Extreme gradient boosting de beste resultaten gaf, met een accuracy van 78\%, gevolgd door Adaptive Boosting (77\%) en Random Forest (75\%). 

\subsection{Adaptive Boosting}
\label{sub:adaboost}
De Adaptive Boost, of AdaBoost, is typisch een classificatie tussen twee klassen  \autocite{Guo2001}. Deze ML-techniek kan dus toegepast worden op de voorspelling van man of vrouw (0 of 1). Om het herkenningsprobleem met meerdere klassen op te lossen, kan een majority voting (MV) strategie worden gebruikt om alle paarsgewijze classificatieresultaten te combineren. Hierbij kiezen we de meest voorkomende klasse als voorspelling. AdaBoost is een adaptief algoritme om een reeks classificeerders te boosten, in die zin dat de gewichten dynamisch worden bijgewerkt op basis van de fouten in eerdere leerresultaten.

\subsection{Random Forest Classifier}
\label{sub:randomforest}
Random forests zijn verzamelingen decision trees die elk getraind zijn op een willekeurig gekozen deelverzameling van de beschikbare data. Deze decision trees worden willekeurig door de trainingsvoorbeelden die aan elke boom worden gegeven,  maar ook door een willekeurige subset van tests die beschikbaar zijn voor optimalisatie op elke node \autocite{Fanelli2012}. Wanneer we een input vector aan het model geven, gaat het model bewegen door de gepaste boom \autocite{Chen2011}. De uiteindelijke classificatie is gebaseerd op een majority voting over alle bomen. Uit het onderzoek van \textcite{Wang2010} blijkt dat Random Forest ook geschikt is voor feature selectie. \\
\\
Een voorbeeld voor het gebruik van Random Forest in gezichtsanalyse is figuur {~\ref{fig:randomforest}}. Hierbij wordt een boom getoond voor het  schatten van de houding van het hoofd. 
\begin{figure}
    \centering
    \includegraphics{graphics/headposition.png}
    \caption[Random forest voor houding van het hoofd]{\label{fig:randomforest}Voorbeeld van een random forest voor het schatten van de houding van het hoofd\autocite{Fanelli2012}.}
\end{figure}
\\
Het onderzoek van \textcite{Mady2018} gebruikt random forest classifier voor de herkenning en detectie van gezichten. De classifier wordt toegepast op het gezicht om dit te herkennen uit de databank. Hier wordt aangehaald wanneer we het aantal bomen in de random forest verhogen, zal ook de nauwkeurigheid verhogen, maar is het ook veel tijdrovender. De random forest is dus een trade-off tussen accuracy en tijd. Het toevoegen van Local Binary Pattern (LBP) en Histogram of Oriented Gradient(HOG) verhoogde de algemene accuracy van de classifier. 

\subsection{Support Vector Machines} \label{sub:svm}
Een Support Vector Machine (SVM) is een ML-techniek die gebruikt wordt voor patroonclassificatie en regressieanalyse \autocite{Chen2011}. Het is gebaseerd op het zoeken naar een lineaire grens tussen twee klassen van patronen.Het SVM mechanisme gaat opzoek naar de optimale classifier die de data in 2 verschillende klassen opdeelt \autocite{Rustam2018}. De optimale classifier wordt de hyperplane genoemd. Dit kunnen we vinden door de marge te maximaliseren en zal zorgen voor zo min mogelijk misclassificatie. \\
\\
Voor niet-lineair scheidbare data mappen we alle punten naar een feature ruimte door gebruik te maken van een kernel \autocite{Shah2012}. Na het splitsen, kunnen we de punten terug mappen naar de input ruimte met een kromme hyperplane, zoals op figuur {~\ref{fig:svm}}. \\
Een SVM is zeer effectief wanneer een hoog aantal dimensionele ruimtes zijn. Ze zijn ook veelzijdig, aangezien we meerdere kernels kunnen gebruiken. De keuze van kernel is afhankelijk van de requirements van het model. \\
\\
In het onderzoek van \textcite{Rustam2018} wordt SVM gebruikt voor de classificatie van leeftijdsgroepen en bereikt de maximale accuracy van 100 \%.
\begin{figure}
    \centering
    \includegraphics{graphics/svm-non-lineair.PNG}
    \caption[SVM]{\label{fig:svm}Support Vector Machine voor niet-lineair scheidbare data}
\end{figure}

\subsection{Ensemble}
\label{sub:ensemble}
Een set van 30 Lineaire Discriminant Analyse (LDA) dienen als basis classifiers en dragen bij tot het ensemble in het model van \textcite{Khan2017}. Het ensemble gebruikt de gecombineerde classificatiecapaciteiten van de base classificiers, zo verbeteren we de algehele prestaties. De volledige pipeline voor de gezichtsanalyse vinden we in figuur {~\ref{fig:ensemble}}. 
\begin{figure}
    \centering
    \includegraphics[width=\columnwidth]{graphics/ensemble.png}
    \caption[Robuust gezichtsherkenning systeem]{\label{fig:ensemble}Robuust Gezichtsherkenning systeem\autocite{Khan2017}.}
\end{figure}

\subsection{Multi attribution model}\label{sub:mamodel}
In het onderzoek van \autocite{Gupta2022} wordt een multi-attribution model uitgewerkt. Dit voorspelt leeftijd en geslacht door middel van slechts 1 model. Het multi-attribution model voorspelt dus beide: leeftijd en geslacht. Het model van \autocite{Guo2014}, dat gebruik maakt van de Biologically Inspired Feature (BIF), gaf betere resultaten in vergelijking met individuele feature modellen. Een CNN is het meest gebruikte model voor multi-attribution, maar dit valt buiten de scope van deze bachelorproef.

\section{Dataset} \label{sec:dataset}
Uit onderzoek van \textcite{Karkkainen2021} bleek dat bestaande, publieke datasets sterk bevoordeeld zijn naar blanke mensen. Modellen die getraind worden op enkel 1 publieke dataset vertonen dan ook slechte en inconsistente resultaten. Kärkkäinen gebruikt in zijn onderzoek een zelf samengestelde dataset, met 108501 afbeeldingen die verdeeld zijn over elk ras. De resultaten van het onderzoek op de dataset vertonen een betere accuraatheid over verschillende rassen en leeftijdsgroepen. \\
\\
Ook in het onderzoek van \textcite{Buolamwini2018} wordt een eigen dataset aangemaakt. Hier wordt niet enkel gekeken naar ras, maar bijvoorbeeld ook naar dikte en hoeveelheid haar. Het model moet zodoende dezelfde accuraatheid geven over verschillende leeftijdsgroepen, geslachten, rassen, et cetera. Uit het onderzoek bleek ook dat resultaten bij modellen die gebruik maken van Microsoft's Face Detect of Face++ een slechtere precisiescore vertonen, mogelijks door bias in trainingsdata.

\section{Voorspellen van geslacht en leeftijd} \label{sec:voorspellen}
Het eerste baanbrekende onderzoek op het vlak van classificatie op basis van gezichtsafbeeldingen dateert uit 1994. \textcite{Kwon1994} voorspelde slechts 3 leeftijdsgroepen: baby's, tieners en volwassen. Het onderzoek focuste zich op het vinden van de primaire features in het gezicht, zoals ogen, mond en neus, en combineerde dit met secundaire features, zoals rimpels. Het onderzoek toonde hiermee aan dat de gezichtsfeatures een grote rol spelen in het classificeren van leeftijd. \\
\\ 
Menselijke gezichten zijn onderhevig aan veroudering en groei \autocite{Gupta2022}. Deze verandering in uiterlijk kan verschillen van persoon en is het gevolg van verschillende factoren, zoals gezondheid, levensstijl, ras, roken, et cetera. \\
Vrouwen en mannen verouderen anders, omdat ze een verschillend verouderingspatroon in het gezicht hebben. Dit is afkomstig van het gebruik van makeup, haarstijl, accesoires van vrouwen of snor en baard bij mannen. Vrouwelijke gezichten ogen dan ook vaak jonger dan mannelijke gezichten. \\
\\
Het voorspellen van leeftijd kan ingedeeld worden in 2 categorieën: leeftijdsgroep classificatie, waarbij we de leeftijd opdelen in ranges (bijvoorbeeld 10-18 jaar) of als regressieprobleem, waarbij we een exact getal gaan voorspellen \autocite{Gupta2022}. In de proof-of-concept van deze bachelorproef zal een leeftijdsgroep worden voorspeld. \\

\section{Performantiescores} \label{sec:performantiescores}
Er kunnen verschillende evaluaties uitgevoerd worden om de performantie van het model te testen \autocite{Sanil2023}. Er kan hiervoor een confusion matrix worden opgesteld, zoals weergegeven in figuur {~\ref{fig:confusionmatrix}}. Deze bevat de True Positives (TP), True Negatives (TN), False Positives (FP) en False Negatives (FN), waarbij TP en TN de correcte voorspellingen weergeven en FP en FN de incorrecte voorspellingen. De confusion matrix werkt voor binaire voorspellingen, ideaal voor de voorspelling van geslacht. Het onderzoek van \textcite{Sanil2023} berekent op basis van de opgestelde matrix 5 scores.
 
\begin{figure}
    \centering
    \includegraphics{graphics/confusionmatrix.PNG}
    \caption[Confusion matrix]{\label{fig:confusionmatrix}Confusion matrix\autocite{Narkhede2018}.}
\end{figure}
\begin{enumerate}
    \item De accuracy geeft weer hoe goed het model voorspellingen maakt. Dit wordt voorgesteld door het aantal correcte voorspellingen  \autocite{Narkhede2018}.
    \[ \text{Accuracy} = \frac{\text{TP+TN}}{\text{TP+TN+FP+FN}} \]
    
    \item De precision stelt de hoeveelheid correcte positieve voorspellingen voor.
    \[ \text{Precision} = \frac{\text{TP}}{\text{TP+FP}} \]
    
    \item De recall, ook wel sensitivity genoemd, stelt de kans op een positieve klasse voor.  
    \[ \text{Recall} = \frac{\text{TP}}{\text{TP+FN}} \]
    
    \item De specificity, stelt de kans op een negatieve klasse voor.
    \[ \text{Specificity} = \frac{\text{TN}}{\text{TN+FP}} \]
    
    \item De F1-score is het gemiddelde van de precision en recall scores, waarbij 1.0 de maximale score is.
    \[ \text{F1-score} = \frac{\text{2 x Precision x Recall}}{\text{Precision + Recall}} \]

\end{enumerate}
    %%=============================================================================
%% Methodologie
%%=============================================================================

\chapter{\IfLanguageName{dutch}{Methodologie}{Methodology}}%
\label{ch:methodologie}

%% TODO: In dit hoofstuk geef je een korte toelichting over hoe je te werk bent
%% gegaan. Verdeel je onderzoek in grote fasen, en licht in elke fase toe wat
%% de doelstelling was, welke deliverables daar uit gekomen zijn, en welke
%% onderzoeksmethoden je daarbij toegepast hebt. Verantwoord waarom je
%% op deze manier te werk gegaan bent.
%% 
%% Voorbeelden van zulke fasen zijn: literatuurstudie, opstellen van een
%% requirements-analyse, opstellen long-list (bij vergelijkende studie),
%% selectie van geschikte tools (bij vergelijkende studie, "short-list"),
%% opzetten testopstelling/PoC, uitvoeren testen en verzamelen
%% van resultaten, analyse van resultaten, ...
%%
%% !!!!! LET OP !!!!!
%%
%% Het is uitdrukkelijk NIET de bedoeling dat je het grootste deel van de corpus
%% van je bachelorproef in dit hoofstuk verwerkt! Dit hoofdstuk is eerder een
%% kort overzicht van je plan van aanpak.
%%
%% Maak voor elke fase (behalve het literatuuronderzoek) een NIEUW HOOFDSTUK aan
%% en geef het een gepaste titel.
Het onderzoek start met het opstellen van een duidelijke requirementsanalyse. Aan de belanghebbenden van IntelliProve wordt nagevraagd aan welke criteria het model moet voldoen. Deze requirements zijn van belang om de deelvragen voor het onderzoek op te stellen. Deze deelvragen kunnen teruggevonden worden in {~\ref{ch:inleiding}}. \\
\\
Na het definiëren van de requirements kan een uitgebreide literatuurstudie {~\ref{ch:standvanzaken}} worden uitgevoerd. Hierin wordt allerhande informatie verzameld over technieken en modellen. Na het uitvoeren van de literatuurstudie wordt duidelijk beeld gevormd van welke modellen er bestaan en verwerkt kunnen worden in de Proof of Concept. Er wordt ook een flowchart opgesteld van de pipeline. Hieruit kunnen duidelijk alle stappen gehaald worden om de Proof of Concept stapsgewijs te doen verlopen. Deze flowchart kan gevonden worden op figuur {~\ref{fig:pipeline}}\\
\\ 
Uit de informatie en kennis van de literatuurstudie kan het onderzoek worden uitgevoerd als Proof of Concept {~\ref{ch:proofofconcept}. Alle verschillende modellen en technieken kunnen uitgetest worden en geëvalueerd op de dataset.  Het resultaat van de Proof of Concept is een pipeline die op basis van een gezichtsfoto het geslacht en de leeftijd van een persoon gaat voorspellen. \\
\\
Alle resultaten van het onderzoek en eventueel verder onderzoek worden besproken in de Conclusie {~\ref{ch:conclusie}}.

\section{Requirements} \label{sec:meth-requirements} 
\begin{itemize}
    \item De pipeline moet leeftijd en geslacht voorspellen op basis van een frontale gezichtsfoto. Er kan gewerkt worden met 1 machine learning model die beide voorspelt of 2 verschillende modellen die gecombineerd worden tot 1 voorspelde waarde. 
    \item De leeftijd wordt in ranges voorspeld, het gaat dus niet om een exact getal. 
    \item Het geslacht wordt voorspeld op basis van het biologische geslacht, man of vrouw.
    \item De trainings- en test dataset moeten online beschikbaar gesteld zijn. Er is geen interne dataset ter beschikking (omwille van privacy-wetgeving).
    \item De pipeline bestaat uit preprocessing van de afbeelding en het voorspellen van leeftijd en geslacht.
    \item Er is geen opgegeven streefdoel voor de accuraatheid van het model. Dit moet zo hoog mogelijk liggen, maar de conclusies die hieruit getrokken kunnen worden zijn van het grootste belang. Andere modellen kunnen voorgesteld worden, de score kan verklaard worden en toekomstig onderzoek kan aangeboden worden. 
\end{itemize}
\begin{figure}
    \centering
    \includegraphics{graphics/pipeline.PNG}
    \caption[Stappen in de pipeline voor Proof of Concept]{\label{fig:pipeline} Stappen in de pipeline voor Proof of Concept}
\end{figure}
 
    
    % Voeg hier je eigen hoofdstukken toe die de ``corpus'' van je bachelorproef
    % vormen. De structuur en titels hangen af van je eigen onderzoek. Je kan bv.
    % elke fase in je onderzoek in een apart hoofdstuk bespreken.
    
    %%=============================================================================
%% Proof of concept
%%=============================================================================

\chapter{Proof of Concept}%
\label{ch:proofofconcept}

\section{Dataset} \label{sec:poc-dataset}
Uit de voorgaande literatuurstudie in hoofdstuk ~\ref{sec:dataset} kan geconcludeerd worden dat de FairFace dataset van \textcite{Karkkainen2021} de meest geschikte is voor gezichtsanalyse. Deze dataset wordt beschikbaar gesteld via Google Drive of Kaggle. De dataset is beschikbaar met een padding (marge) van 0.25 of 1.25. Voor deze Proof of Concept wordt de dataset met een padding van 0.25 gebruikt. Deze is het meest geschikt voor wetenschappelijk onderzoek, omdat deze door de kleinere marge minder opslag verbruikt en het volledige gezicht weergeeft. De dataset is verder opgedeeld in 86.744 train afbeeldingen en 10.954 validatie of test afbeeldingen. Dit maakt de FairFace dataset een bijzonder grote dataset met kleine bias en beperkte opslag, wat ideaal is voor wetenschappelijk onderzoek. \\
\\
Doordat de zip folder te groot is om direct te downloaden en steeds een waarschuwing genereert via Google Drive, moet de dataset gedownload en uitgepakt worden. Dit kan manueel, of efficiënter in Python via codevoorbeeld ~\ref{sc:dataset} . De dataset kan dus niet in memory opgeslagen worden of rechtstreeks in Google Drive worden aangesproken.\\
\\
De labels van de trainingsset bevinden zich in een csv bestand (in dezelfde Google Drive). Om uit te testen of de dataset correct werd ingeladen, is het mogelijk om enkele willekeurige afbeeldingen weer te geven met hun leeftijdscategorie en geslacht label. In deze Proof of Concept wordt dan ook een leeftijdscategorie voorspelt en geen leeftijdsgetal. Dit zorgt ervoor dat het voorspellen van de leeftijd een classificatietaak wordt, net zoals het voorspellen van geslacht. Wanneer er een leeftijdsgetal wordt voorspeld, is dit een regressieprobleem \autocite{Geron2019}. Een voorbeeld van de uitvoer van deze test wordt weergegeven op figuur~\ref{fig:5randomtrain}. Voor deze Proof of Concept worden enkel de labels voor de leeftijd en geslacht gebruikt. Daarnaast bevat de FairFace dataset ook een label voor ras.

\begin{lstlisting}[style=mystyle, caption={Functie om FairFace dataset te downloaden en uitpakken \autocite{Ramos2024}}, label={sc:dataset}]
# Download the dataset from Google Drive and save in the data directory
# URL to the Google Drive file
url = 'https://drive.google.com/u/0/uc?id=1Z1RqRo0_JiavaZw2yzZG6WETdZQ8qX86&export=download'

zip_file = gdown.download(url, quiet=False)

destination_dir = './data'
os.makedirs(destination_dir, exist_ok=True) # Create the directory if it doesn't exist

# Extract the zip file into the specified directory
with zipfile.ZipFile(zip_file, 'r') as z:
z.extractall(destination_dir)
\end{lstlisting}



\begin{figure}
    \centering
    \includegraphics[width=\columnwidth]{graphics/randomtrain.png}
    \caption[5 willekeurige train voorbeelden]{5 willekeurige train voorbeelden met leeftijd en geslacht label}
        \label{fig:5randomtrain}}
\end{figure}

\section{Feature extractie} \label{sec:poc-featextractie}
De feature extractie is de eerste preprocessing stap in de machine learning pipeline. In deze Proof of Concept wordt het getrainde dlib model van \textcite{King2024} gebruikt. Dit model detecteert en voorspelt de 68 landmarks uit het gezicht. Het model is beschikbaar via GitHub en is beperkt in opslag. Doordat het model al voorgetraind is, hoeft dit dus niet opnieuw getraind te worden en versnelt het preprocessing proces. Door het detecteren van de landmarks kunnen we deze behouden en de afbeelding normaliseren, zoals in {~\ref{sub:normalisatie}}. \\
\\
\textcite{King2024} stelt ook een model ter beschikking waarop slechts 5 landmarks gedetecteerd worden. Dit zijn de 2 uiteinden van de ogen en het midden van de neus. Figuur ~\ref{fig:5landmarks} geeft een voorbeeld van een gezichtsafbeelding waarop deze 5 landmarks worden aangeduid. Voor deze use case zijn de 5 landmarks duidelijk te weinig om de volledige vorm van het gezicht weer te geven. Dit model met 5 landmarks is efficiënter bij een simpele gezichtsdetectie, om opslag te besparen en het model te versnellen.

\begin{figure}[H]
    \centering
    \includegraphics[width=\columnwidth]{graphics/5landmarks.PNG}
    \caption[5 landmarks op het gezicht]{5 landmarks op het gezicht}
    \label{fig:5landmarks}}
\end{figure}

\begin{enumerate}
    \item De eerste stap in dit proces omvat het detecteren van de 68 landmarks  \autocite{Serengil2020}. Deze omvatten de vorm van het gezicht, wenkbrauwen, ogen, neus en mond. Dit wordt uitgevoerd met het \textit{shape\_predictor\_68\_face\_landmarks} model van \textcite{King2024}. Vervolgens wordt de \textit{circle} functie van \textcite{OpenCV2024} gebruikt om de landmarks aan te duiden op de afbeelding. \\
    Figuur {~\ref{fig:68landmarks} geeft het resultaat van deze stap weer. \\
    \item Vervolgens wordt op basis van de gedetecteerde landmarks de vorm van het gezicht uit de afbeelding gehaald. De coördinaten van deze landmarks worden opgehaald en met \textit{line} van \textcite{OpenCV2024} omcirkeld. Er wordt een lijn getrokken van het eerste punt naar het volgende in de lijst met coördinaten. Figuur {~\ref{fig:gezichtsvorm}} geeft het resultaat van deze stap. \\ 
    \item Ten slotte maken we een mask voor de overgebleven afbeelding. Alles rond de vorm van het gezicht wordt met \textit{fillConvexPoly} van \textcite{OpenCV2024} zwart ingekleurd. Hierdoor wordt de onbelangrijke achtergrond van de afbeelding verwijdert en wordt enkel het gezicht overgehouden. Deze stap voert de normalisatie net zoals \textcite{Chen2011} uit, gebruikmakend van de HOG features in het detecteren van de landmarks, omschreven in  ~\ref{sub:hog}. Figuur {~\ref{fig:normalisatie}} geeft het resultaat van deze stap weer. Dit is een genormaliseerde afbeelding die gebruikt zal worden voor de feature dimensionaliteitsreductie.
\end{enumerate}\\

Het normaliseren van de afbeeldingen kan in een functie worden gestoken. Hierdoor kan aan de volledige dataset een mask toegevoegd worden, in plaats van slechts 1 afbeelding voordien. Het toevoegen van een mask aan een afbeelding duurt minder dan een seconde. Dit maakt deze methode geschikt voor de grote dataset.  Het toevoegen van de mask wordt in batches gedaan, om het proces te versnellen. Er wordt hierdoor steeds aan een aantal afbeeldingen tegelijk een mask toegevoegd, in plaats van 1 afbeelding per keer. Er wordt gekozen voor een batch size van 128, omdat deze voldoende groot is voor de beschikbare rekenkracht op een standaard laptop. Het volledige proces nam 84 minuten in beslag voor 86.744 afbeeldingen. \\
\\
Daarnaast kan deze functie ook valideren dat er geen duidelijke gezichten aanwezig zijn op de afbeeldingen, doordat de 68 landmarks niet gedetecteerd kunnen worden. Deze afbeeldingen zijn onbruikbaar en worden zo ook uit de dataset gehaald. Figuur {~\ref{fig:noface}} geeft een afbeelding weer waarop de landmarks niet gedetecteerd kunnen worden. Dit komt doordat de foto geen vooraanzicht is en op deze figuur zijn beide ogen niet duidelijk weergegeven.  Codevoorbeeld {~\ref{sc:mask} geeft deze functie weer \autocite{Serengil2020}. In de genormaliseerde trainingsdataset blijven nog 63.938 afbeeldingen over, in de genormaliseerde testset blijven nog 8.028 over. Beide datasets verliezen na normalisatie zo ongeveer 27\% van de gezichtsafbeeldingen, maar zijn nog voldoende groot om mee verder te werken.

\begin{figure}[H]
    \centering
    \includegraphics{graphics/68landmarks.png}
    \caption[68 landmarks op gezicht]{68 landmarks op het gezicht}
    \label{fig:68landmarks}}
\end{figure}

\begin{figure}[H]
    \centering
    \includegraphics{graphics/gezichtsvorm.png}
    \caption[Gezichtsvorm uit afbeelding halen]{Gezichtsvorm uit afbeelding halen op basis van landmarks}
    \label{fig:gezichtsvorm}}
\end{figure}

\begin{figure}[H]
    \centering
    \includegraphics{graphics/normalisatie.png}
    \caption[Normalisatie van gezichtsafbeelding]{Normalisatie van gezichtsafbeelding}
    \label{fig:normalisatie}}
\end{figure}

\begin{lstlisting}[style=mystyle, caption={Functie om mask toe te voegen aan de volledige dataset \autocite{Serengil2020}}, label={sc:mask}]
import os
import cv2
import dlib
import numpy as np

def get_mask(image_paths, predictor_path, output_folder):
# Initialize face detector and landmark detector
face_detector = dlib.get_frontal_face_detector()
landmark_detector = dlib.shape_predictor(predictor_path)

for image_path in image_paths:
image = dlib.load_rgb_image(image_path)

# Detect faces in the image -> do not mask images where no face is found
faces = face_detector(image, 1)
no_faces = []
if len(faces) == 0:
print("No faces found in the image:", image_path)
no_faces.append(image_path)
continue

# Predict facial landmarks
face = faces[0] # only one face
landmarks = landmark_detector(image, face)
landmarks_tuple = [(landmarks.part(i).x, landmarks.part(i).y) for i in range(68)]

# Define the route for the mask
routes = [i for i in range(16,-1,-1)] + [i for i in range(17,19)] + [i for i in range(24,26)] + [16]
routes_coordinates = [landmarks_tuple[i] for i in routes]

# Create a mask
mask = np.zeros((image.shape[0], image.shape[1]), dtype=np.uint8)
mask = cv2.fillConvexPoly(mask, np.array(routes_coordinates), 255)
masked_image = cv2.bitwise_and(image, image, mask=mask)

# Save the masked image to the output folder with the same filename
filename = os.path.splitext(os.path.basename(image_path))[0]
output_filename = filename + "_masked.jpg"
output_path = os.path.join(output_folder, output_filename)
cv2.imwrite(output_path, masked_image)

print("Masked image saved:", output_path)

# Configuration
predictor_path = "./shape_predictor_68_face_landmarks.dat"
input_folder = 'data/train'
output_folder = './data/masked_train'
batch_size = 128

# Create the output folder if it doesn't exist
if not os.path.exists(output_folder):
os.makedirs(output_folder)

# List all image files in the input folder
image_files = sorted([f for f in os.listdir(input_folder) if f.endswith(('.jpg', '.jpeg', '.png'))], key=lambda x: int(os.path.splitext(x)[0]))

# Process images in batches
for i in range(0, len(image_files), batch_size):
batch_images = image_files[i:i+batch_size]
batch_paths = [os.path.join(input_folder, image) for image in batch_images]
get_mask(batch_paths, predictor_path, output_folder) 
\end{lstlisting} 
\begin{figure}
    \centering
    \includegraphics{graphics/no_face_detected.png}
    \caption[Geen gezicht gedecteerd op afbeelding]{De 68 landmarks werden niet teruggevonden op de gezichtsafbeelding, deze wordt verwijderd uit de dataset.}
    \label{fig:noface}}
\end{figure}

% ---- Feature dim
\section{Feature dimensionaliteitsreductie}\label{sec:poc-featuredim}
Op de genormaliseerde afbeeldingen wordt Truncated SVD toegepast als dimensionaliteitsreductie van de features. Doordat de dataset zeer veel afbeeldingen bevat, wordt geopteerd voor Truncated SVD in plaats van het gekende PCA. PCA vereist een enorme rekenkracht op een grote dataset, omdat deze de covariantie van de matrix van de originele dataset berekent. Truncated SVD berekent de singular value decomposition van de datamatrix, wat veel efficiënter is voor grote en sparse datasets \autocite{Baruah2023}. De benodigde opslag zal verminderen en de impact van noise op de dataset daalt. Het uitvoeren van PCA op de dataset, of zelfs een subset daarvan, genereert meteen een foutmelding over de benodigde opslag. Een standaard laptop kan deze berekening niet aan.\\
\\
De Truncated SVD functie wordt beschikbaar gesteld via \textit{sklearn.decomposition}. Aan Truncated SVD wordt het aantal componenten, dat behouden moet worden, als paramater meegegeven. Dit is de gewenste dimensionaliteit van de output data. De dimensionaliteit moet steeds minder zijn dan het aantal features in de data \autocite{ScikitLearn2024}. Het ideale aantal componenten kan berekend worden aan de hand van de verklaarde variantie. Er wordt een lus gemaakt door de Truncated SVD met een range die gaat tot de shape van de trainingsdata. Voor elk aantal componenten wordt de verklaarde variantie berekend. De grafiek met de verklaarde variantie per aantal componenten kan teruggevonden worden op figuur {~\ref{fig:explainedvar}}. \\
\\ 
Er wordt verder gewerkt met 400 componenten, omdat deze een verklaarde variantie van circa 0.99 heeft. Hierdoor daalt het aantal componenten van de data van 1000 naar 400. De features van data worden zo gereduceerd om makkelijker mee verder te werken bij het trainen van de machine learning modellen. Codevoorbeeld {~\ref{sc:truncatedsvd} geeft weer hoe de Truncated SVD wordt toegepast op de volledige trainingsdataset. Het resultaat van de dimensionaliteitsreductie op een gezichtsafbeelding is te vinden op figuur {~\ref{fig:truncated}}.
\begin{figure}[H]
    \centering
    \includegraphics{graphics/explained_variance.png}
    \caption[Verklaarde variantie voor de Truncated SVD]{Verklaarde variantie voor de Truncated SVD}
    \label{fig:explainedvar}}
\end{figure}
\begin{lstlisting}[style=mystyle, caption={Functie om Truncated SVD toe te passen op de volledige dataset \autocite{ScikitLearn2024}}, label={sc:truncatedsvd}]
% Apply TruncatedSVD for dimensionality reduction
svd = TruncatedSVD(n_components=400)
X_train_svd = svd.fit_transform(X_train)  
\end{lstlisting}
\begin{figure}[H]
    \centering
    \includegraphics{graphics/truncatedsvd.png}
    \caption[Truncated SVD toegepast op genormaliseerde gezichtsafbeelding]{Truncated SVD toegepast op genormaliseerde gezichtsafbeelding}
    \label{fig:truncated}}
\end{figure}

\section{Machine learning modellen}\label{sec:poc-mlmodellen}
Om het meest geschikte machine learning model te vinden voor de voorspelling van leeftijd en geslacht op basis van gezichtsafbeeldingen, worden verschillende modellen uit de literatuurstudie in hoofdstuk ~\ref{sec:bestaandeml} getest. De performantie van 2 aparte modellen die leeftijd en geslacht voorspellen, alsook 1 model die beide voorspelt worden geanalyseerd. \\
\\
Om optimaal te zoeken naar de beste parameters wordt gebruik gemaakt van GridSearch ~\ref{sub:gridsearch}. Hieraan worden meerdere parameters meegegeven, zoals aantal estimators of kernel, waarvoor GridSearch de parameter met de beste score na training weergeeft. Aan de parameters van Grid Search wordt 5 als standaardwaarde van de cross validatie meegegeven. Zoals in ~\ref{sub:gridsearch} aangehaald, worden er 5 herhalingen uitgevoerd op de dataset om het model te optimaliseren. De modellen met hun parameters worden geëvalueerd op basis van de balanced accuracy. Deze score gaat het aantal correct voorspelde instanties na, in een dataset die mogelijks ongelijk verdeeld is \autocite{Garcia2009}. Tabel ~\ref{tab:leeftijdcat} geeft weer hoe de leeftijdscategorieën verdeeld zijn en tabel ~\ref{tab:geslacht} hoe geslacht verdeeld is in de trainingsdata. Hieruit kan afgeleid worden dat de leeftijdscategorieën niet even verdeeld zijn en dat balanced accuracy hier wel degelijk de betere optie is. \\
\\
De testfase van de GridSearch wordt uitgevoerd op een subset van 1000 willekeurige afbeeldingen uit de dataset om de tijd die nodig is om te trainen te verminderen. Het model met de parameters die de beste score leveren, wordt dan uiteindelijk getraind op de volledige trainingsdataset. \\
\begin{center}
    \caption{Verdeling van leeftijdscategorieën}
    \label{tab:leeftijdcat}
    \begin{tabular}{||c | c | c||} 
        \hline
        Leeftijdscategorie & Aantal & Percentage van volledige dataset  \\ 
        \hline
        0-2 & 1399 & 2,19\% \\
        \hline
        3-9 & 8003 & 12,52\% \\
        \hline
        10-19 & 6880 & 10,76\% \\
        \hline
        20-29 & 18924 & 29,60\% \\
        \hline 
        30 -39 & 13874 & 21,70\% \\
        \hline
        40-49 & 7726 & 12,08\% \\
        \hline 
        50-59 & 4528 & 7,08\% \\
        \hline 
        60-69 & 1980 & 3,10\% \\
        \hline 
        more than 70 & 624 & 1\% \\
    \end{tabular}
\end{center}

\begin{center}
     \caption{Verdeling van geslacht}
       \label{tab:geslacht}
    \begin{tabular}{||c | c | c||} 
        \hline
        Geslacht & Aantal & Percentage van volledige dataset  \\ 
        \hline
        Male & 32435 & 50,73\% \\
        \hline
        Female & 31503 & 49,27\% \\
        
    \end{tabular}
\end{center}

\\
Om een model te trainen die beide (leeftijd en geslacht) gelijktijdig kan voorspellen, wordt gebruik gemaakt van de MultiOutputClassifier van \textit{sklearn.multioutput}. Deze strategie bestaat uit het aanpassen van een classifier per target. Hiermee kunnen classifiers uitgebreid worden die van nature geen multi-target classificatie ondersteunen, zoals Random Forest Classifier \autocite{ScikitLearn2024}. Het model geeft als output 2 labels, één voor geslacht en één voor de leeftijdscategorie. \\
\\ 
Het getrainde model wordt getest op de validatie dataset. Dit gebeurt met de \textit{predict} functie van Scikit-learn. Op basis hiervan wordt de confusion matrix opgesteld en de performantiescores, aangehaald in de literatuurstudie ~\ref{ch:standvanzaken}, berekent. \textcite{ScikitLearn2024} heeft al enkele ingebouwde functies om de performantiescores, zoals $accuracy\_score$ en $precison\_score$, te berekenen. Deze kunnen uitgevoerd worden op de confusion matrix van het geslacht. Voor de modellen die leeftijd en beide voorspellen dienen de scores handmatig berekend te worden, omdat dit geen 2x2 matrices zijn. 

\subsection{Random Forest Classifier} \label{sub:poc-rfc}
Het Random Forest Classifier model wordt beschikbaar gesteld via \textit{sklearn.ensemble}.
Omdat het model geen tekstuele data kan categoriseren, dienen de labels omgezet te worden naar numerische waarden. Voor het geslacht (Female of Male) wordt dit 0 of 1. Voor de leeftijdscategorieën wordt OrdinalEncoder gebruikt om deze te mappen naar een getal tussen 0 en 8, waarbij '0-2' categorie 0 wordt en 'more than 70' categorie 8 wordt. Aan deze Ordinal Encoder geven we een mapping mee van de labels. De encoder gaat deze dan fitten op de gegeven labels van de dataset volgens de meegegeven mapping \autocite{ScikitLearn2024}.  \\
\\
Aan de Grid Search voor de Random Forest Classifier worden volgende parameters meegegeven:
\begin{itemize}
    \item $n\_estimators$: het aantal bomen dat gebruikt wordt (default = 100). 
    \item $max\_features$: het aantal features die gebruikt worden om de beste scheiding in de bomen te vinden (default = 'sqrt')
    \item $min\_samples\_split$: het minimum aantal instanties die nodig zijn om node te mogen splitsen (default = 2).
    \item $min\_samples\_leaf$: minimum aantal instanties die in een blad moeten zitten (default = 1). Een blad is het uiteinde van een boom, die de voorspelde klasse weergeeft \textcite{Garg2020}. 
    \item $max\_depth$: de maximale diepte van een boom (default = None). 
    \item bootstrap: het gebruik van steekproeven bij het bouwen van de bomen (default = True)
\end{itemize}
\\
\begin{center}
    \begin{tabular}{||c | c  | c | c||} 
        \hline
          & Geslacht & Leeftijd & Beide  \\ 
        \hline
        Parameters & $n\_estimators=230$ & $max\_depth$=5, $n\_estimators$=200 & $n\_estimators$=100  \\ 
        \hline
        Accuracy & 99,99\% & 29,71\% & 99,95\%  \\
        \hline
        Precision & 99,99\% & 0,00\% & 99,93\%  \\
        \hline
        Recall & 100\% & 11,21\% & 99,92\%  \\
        \hline
        Specificity & 99,99\% & 0,00\% & 99,99\% \\
        \hline
        F1 & 99,99\% & 0.00\% & 99,93\%  \\
    \end{tabular}
\end{center}
\\
\\
Figuur ~\ref{fig:cmrfc} geeft de confusion matrices voor de 3 modellen weer. Uit de confusion matrices en de scores kunnen we afleiden dat de Random Forest Classifier uitstekend presteert voor de voorspelling van geslacht en beide. Voor de voorspelling van de leeftijdscategorie behaalt dit model slechts een score van circa 30\%. Uit de confusion matrix van de leeftijdsvoorspellingen is ook af te leiden dat vrijwel enkel de leeftijdscategorie 20-29 wordt voorspelt. Dit kan te verklaren zijn door overfitting, omdat deze klasse het meest aanwezig is in de dataset is de kans het grootst dat de voorspelling juist is \autocite{Kreiger2020}. Het model presteert wel goed wanneer het geslacht ook wordt toegevoegd. Dit kan te verklaren zijn door de levensfactoren en opvoeding, aangehaald in ~\ref{sec:voorspellen}. Het model krijgt de context van het geslacht mee en kan hierdoor wel patronen herkennen. \\
\begin{figure}[H]
    \centering
    \includegraphics{graphics/cm_rfc.png}
    \caption[Confusion Matrix voor Random Forest Classifier]{Confusion Matrix voor Random Forest Classifier}
    \label{fig:cmrfc}}
\end{figure}

\subsection{XGBoost Classifier} \label{sub:poc-xgb}
Het XGBoost, Extreme Gradient Boosting, Classifier model wordt beschikbaar gesteld via \textit{xgboost}. Ook aan dit model kan geen tekstuele data worden meegeven en worden de labels omgezet in numerische waarden, zoals in ~\ref{sub:poc-rfc}. \\
\\
Voor de classificatie van beide labels met de MultiOutputClassifier vindt nog een preprocessing stap plaats. De tekstuele labels moeten omgezet worden naar numerische categorieën. Hiervoor wordt de LabelEncoder van \textit{sklearn.preprocessing} gebruikt. De labels worden dan samengevoegd met \textit{$np.column\_stack$}. Codevoorbeeld ~\ref{sc:prepmulti} geeft weer hoe de labels worden aangepast voor de MultiOuputClassifier. Het label voor een vrouwelijke 0-2 jarige wordt zo [0,1]. \\
\\
Aan de Grid Search voor de XGBoost Classifier worden volgende parameters meegegeven: 
\begin{itemsize}
    \item $n\_estimators$: het aantal bomen dat gebruikt wordt (default = 100). 
    \item $max\_depth$: de maximale diepte van een boom (default = 6). 
\end{itemsize}

\begin{lstlisting}[style=mystyle, caption={Functie om labels aan te passen voor de MultiOutputClassifier \autocite{Roepke2024} en \autocite{Numpy2024}}, label={sc:prepmulti}]
        age_class_encoded = label_encoder.fit_transform(y_train_age)
        gender_encoded = label_encoder.fit_transform(y_train_gender)
        y_combined = np.column_stack((age_class_encoded, gender_encoded))
    \end{lstlisting}

\begin{center}
    \begin{tabular}{||c | c | c | c||} 
        \hline
        & Geslacht & Leeftijd & Beide  \\ 
        \hline
        Parameters & $n\_estimators=267$, $depth$=5 &  $n\_estimators$=300, $depth$=4 & $n\_estimators$=100 \\ 
        \hline
        Accuracy & 94.76\% & 81,82\% & 10,23\%  \\
        \hline
        Precision & 95.22\% & 90,98\% & 6,50\%  \\
        \hline
        Recall & 94.09\% & 84,16\% & 5,35\% \\
        \hline
        Specificity & 95.41\% & 97,30\% & 94,32\% \\
        \hline
        F1 & 94.65\% & 87,44\% & 5,87\%  \\
      
    \end{tabular}
\end{center}
\\
\\
Figuur ~\ref{fig:cmxgb} geeft de confusion matrices voor de XGBoost classifier modellen weer. Hieruit kunnen we afleiden dat de XGBoost Classifier zeer goed presteert voor de aparte voorspelling van leeftijd en geslacht. Wanneer beide labels gelijktijdig worden voorspeld, presteert het model zeer slecht. Ook dit kan te verklaren zijn door overfitting, de aard van de data of het model kan geen patronen vinden. 

\begin{figure}[H]
    \centering
    \includegraphics[width=\columnwidth]{graphics/cm_xgb.png}
    \caption[Confusion Matrix voor XGBoost Classifier]{Confusion Matrix voor XGBoost Classifier}
    \label{fig:cmxgb}}
\end{figure}
\subsection{Support Vector Machines} \label{sub:poc-svm}
Het Support Vector Classifier (SVC) model wordt beschikbaar gesteld via \textit{sklearn.svm}. Een SVM kan wel classificeren op basis van tekstuele data. Aan dit model wordt wel de originele dataset meegegeven en worden de labels niet gewijzigd. \\
\\
Aan de Grid Search voor het SVC model worden volgende parameters meegegeven:
\begin{itemize}
    \item C: regularisatie parameter, afhankelijk van de grootte van de dataset (default = 1.0).
    \item gamma: coëfficiënt van de kernel (default = 'scale')
    \item kernel: kernel type die gebruikt wordt in het algoritme (default = 'rbf').
    \item degree: graad van de kernel (default = 3).
\end{itemize}
\begin{center}
    \begin{tabular}{||c | c | c | c||} 
        \hline
        & Geslacht & Leeftijd & Beide  \\ 
        \hline
        Parameters & C=125 &  C=125, kernel='linear', degree=1 & C=25  \\ 
        \hline
        Score model (train) & 68.41\% & 21.52\% & 20.50\%  \\
        \hline
        Accuracy & 66.79\% & 28.23\% & 21.55\%  \\
        \hline
        Precision & 67.57\% & 22.12\% & 19.44\%  \\
        \hline
        Recall & 66.40\% & 21.70\% & 13.87\%  \\
        \hline
        Specificity & 67.19\% & 90.17\% & 95.15\% \\
        \hline
        F1 & 66.98\% & 21.91\% & 16.18\%  \\
       
    \end{tabular}
\end{center}
\\
\\
Figuur ~\ref{fig:svc} geeft de confusion matrices voor de SVC modellen weer. Uit de scores kan afgeleid worden dat het SVC model slechter presteert dan voorgaande modellen. Voor de voorspelling van geslacht wordt hier slechts een score van circa 70\% behaald. Ook de modellen voor de voorspelling van leeftijd en beide presteren slecht, met een score van slechts 20\%. Ook uit de confusion matrices is af te leiden dat er veel foute voorspellingen zijn. Hieruit kan geconcludeerd worden dat een SVC niet het geschikte model is voor de classificatie van gezichtsafbeeldingen. 
\\
\begin{figure}[H]
    \centering
    \includegraphics[width=\columnwidth]{graphics/cm_svc.png}
    \caption[Confusion Matrix voor Support Vector Classifier]{Confusion Matrix voor Support Vector Classifier}
    \label{fig:svc}}
\end{figure}
\subsection{Voting Classifier} \label{sub:poc-votingclass}
De Voting Classifier is een ensemble techniek die meerdere modellen combineert. De SVC, random forest en XGBoost Classifier worden tegelijk aangesproken om de accuracy van het model te verhogen. De parameters die het beste scoorden voor de afzonderlijke modellen worden hier hergebruikt. Codevoorbeeld ~\ref{?} geeft weer hoe de Voting Classifier de verschillende modellen combineert. 

\begin{center}
    \begin{tabular}{||c c c c||} 
        \hline
        & Geslacht & Leeftijd & Beide  \\ 
        \hline
        Parameters & ? &  ? & ?  \\
        \hline
        Accuracy & ?\% & ?\% & ?\%  \\
        \hline
        Precision & ?\% & ?\% & ?\%  \\
        \hline
        Recall & ?\% & ?\% & ?\%  \\
        \hline
        Specificity & ?\% & ?\% & ?\% \\
        \hline
        F1 & ?\% & ?\% & ?\%  \\
        
    \end{tabular}
\end{center}
    %...
    
    \input{conclusie}
    
    %---------- Bijlagen -----------------------------------------------------------
    
    \appendix
    
    \chapter{Onderzoeksvoorstel}
    
    Het onderwerp van deze bachelorproef is gebaseerd op een onderzoeksvoorstel dat vooraf werd beoordeeld door de promotor. Dat voorstel is opgenomen in deze bijlage.
    
    %% TODO: 
    \section*{Samenvatting}
    
    % Kopieer en plak hier de samenvatting (abstract) van je onderzoeksvoorstel.
    Gezichtsanalyse heeft de laatste jaren veel aandacht gekregen vanwege de brede toepassingen op verschillende gebieden, zoals gezondheidszorg, beveiliging en marketing.  
    De focus van de bachelorproef ligt op het ontwikkelen en implementeren van een robuust systeem voor het schatten van leeftijd en geslacht op basis van gezichtsfoto’s, met behulp van machine learning technieken. Dit onderzoek is van bijzonder belang voor het verbeteren van de beoordeling van geestelijke gezondheidszorg door camera-gebaseerde gezondheidsmetingen. Deze inspanningen dragen bij aan de doelstellingen van IntelliProve, een platform dat online gezondheidsoplossingen biedt. Er wordt onderzocht welke machine learning technieken de beste resultaten tonen en uit welke elementen de volledige pipeline zal bestaan.   
    In de eerste fase van het onderzoek wordt een literatuurstudie uitgevoerd om uit bestaand onderzoek de gebruikte modellen te verkennen en hun sterktes, limitaties en performantie te achterhalen. In de proof-of-concept gebeurt eerst de preprocessing, feature extractie en gezichtsdetectie. Daarna worden de modellen uit de literatuurstudie getest op bestaande datasets en wordt een prestatie-evaluatie opgesteld op basis van metrieken zoals accuracy, precision, recall, F1-score en ROC curve. Het verwachte resultaat is een robuuste machine learning pipeline waarmee IntelliProve de leeftijd en het geslacht kan voorspellen op basis van een gezichtsafbeelding. 
    
    % Verwijzing naar het bestand met de inhoud van het onderzoeksvoorstel
    %---------- Inleiding ---------------------------------------------------------

\section{Introductie}%
\label{sec:introductie}

De onderzoeksvraag werd aangeboden door het bedrijf IntelliProve. IntelliProve biedt online gezondheidsoplossingen, een software die in staat is om binnen enkele seconden nauwkeurig gezondheidsparameters te bepalen, gebaseerd op een optische meting van het gezicht.  
Het doel van de bachelorproef is het ontwikkelen en implementeren van een robuust systeem voor het schatten van de leeftijd en het geslacht van personen op basis van gezichtsfoto's, met behulp van machine learning-technieken. 
Dit project is van bijzonder belang voor het verbeteren van de beoordeling van de geestelijke gezondheidszorg door middel van camera-gebaseerde gezondheidsmetingen. Het onderzoek beoogt bij te dragen aan de vooruitgang op dit gebied door gebruik te maken van geavanceerde algoritmen om leeftijd en geslacht nauwkeurig te voorspellen aan de hand van gezichtsbeelden.
De literatuurstudie biedt een inzicht in facial analysis, de bestaande machine learning modellen en hun functionaliteiten. De proof-of-concept zal bestaan uit het ontwikkelen van een machine learning pipeline die in staat is om leeftijd en geslacht te voorspellen op basis van bestaande datasets. De pipeline omvat verschillende image preprocessing technieken om de dataset voor te bereiden op de modeltraining. Om betrouwbaarheid en accuracy te garanderen, worden de modellen verfijnd en geoptimaliseerd om de hoogst mogelijke nauwkeurigheid te bereiken bij het schatten van leeftijd en geslacht.

\section{Stand van zaken}
\label{sec:stand-van-zaken}

Gezichtsanalyse bestaat uit het definiëren van menselijke gezichten in real time, met behulp van computeralgoritmen en machine learning technieken. Voor mensen en computersystemen bevat een gezichtsbeeld details zoals leeftijd, geslacht, stemming, afkomst, et cetera.
Gezichtsanalyse omvat dan ook het lokaliseren en meten van gezichtskenmerken in een afbeelding. Vervolgens wordt het gezichtsbeeld geanalyseerd voor het extraheren van kenmerken \autocite{Sanil2023}. Gezichtsanalyse speelt dan ook een grote rol in real world applicaties, zoals animatie, identiteitsverificatie, medische diagnose, et cetera. Ondanks het bestaande onderzoekswerk over dit onderwerp, is gezichtsanalyse nog steeds een lastige taak vanwege verschillende factoren zoals veranderingen in hoek, gezichtsuitdrukkingen en achtergrond \autocite{Siddiqi2022}.

Volgens onderzoek van \textcite{BasystiukMR23} is beeldherkenning een simpele procedure die slechts uit 3 stappen bestaat. 
\begin{enumerate}
    \item Preprocessing: We voegen filters toe aan de afbeelding om het geschikter te maken voor herkenning.
    \item Feature Extractie: We identificeren belangrijke data en behouden deze om mee verder te werken. Dit wordt verder besproken in \ref{sub:gezichtsdetectie}
    \item Classificatie: het analyseren en identificeren van de  data na de feature extractie.
\end{enumerate}

\subsection{Gezichtsdetectie}
\label{sub:gezichtsdetectie}
Gezichtsdetectie is de eerste stap in het bepalen van de gezichtsfeatures. Deze features zijn de interessante delen van het gezicht , bijvoorbeeld ogen, neus en mond en worden ook wel eens landmarks genoemd \autocite{Coppens2018}. In dit onderzoek raadt de auteur 3 gezichtsdetectieservices aan: Microsoft Face API, Face++ en Kairos. Deze zijn alle 3 ook geschikt om geslacht en leeftijd te voorspellen. Er wordt met 3 services tegelijk gewerkt om elkaars beperkingen op te vangen.
In het onderzoek van \textcite{Sanil2023} berekenen ze de landmarks op 2 verschillende methoden: Euclidische afstand en Geodetische afstand.

\textcite{Sanil2023} stelt 468 landmarks visueel voor, in plaats van 68, om de accuracy te verbeteren. Dit door gebruik te maken van de Mediapipe library (\textcite{Zubair2021}). Op figuur {~\ref{fig:landmarks}} staan de belangrijkste antropometrische landmarks aangeduid . Het is noodzakelijk om de belangrijkste features te identificeren die kunnen leiden tot het correct voorspellen van de leeftijd of het geslacht. Om de leeftijd te voorspellen kunnen we bijvoorbeeld de rimpels op de gezichtsfoto analyseren \autocite{Kwon1994}. 
\begin{figure}
    \centering
    \includegraphics[width=\columnwidth]{graphics/faciallandmarks.png}
    \caption{\label{fig:landmarks}De belangrijkste anthropometrische landmarks\autocite{Sanil2023}.}
\end{figure}

\subsubsection{Normalisatie}
In onderzoek van \autocite{Chen2011} voor het voorspellen van leeftijd op basis van afbeeldingen werden de gezichtsafbeeldingen genormaliseerd voor de feature extraction plaatsvond. De geometrie van de afbeeldingen kan worden genormaliseerd op basis van de gedetecteerde features, zoals oog of mond coördinatie. De auteur maakt gebruik van een mask om de pixels die zich niet in de ovaal van de typsiche gezichtsregio bevinden te verwijderen. Dit gaat dan bijvoorbeeld over haar en hemdskragen. Zo blijven enkel de belangrijkste features van het gezicht over. Figuur {~\ref{fig:beforenormalisation}} geeft een voorbeeld van een dataset met gezichtsfoto's. In figuur {~\ref{fig:afternormalisation}} worden deze afbeeldingen genormaliseerd, waardoor er minder variaties overblijven in de afbeeldingen en we bijvoorbeeld al geen achtergrond overhouden.
\begin{figure}
    \centering
    \includegraphics[width=\columnwidth]{graphics/beforenorm.PNG}
    \caption{\label{fig:beforenormalisation}Gezichtsafbeeldingen voor normalisatie\autocite{Chen2011}.}
\end{figure}
\begin{figure}
    \centering
    \includegraphics[width=\columnwidth]{graphics/afternorm.PNG}
    \caption{\label{fig:afternormalisation}Gezichtsafbeeldingen na het toepassen van normalisatie\autocite{Chen2011}.}
\end{figure}  

\subsection{Bestaande Machine-Learning technieken}
Uit onderzoek naar gezichtsdetectie van \textcite{Sanil2023} bleek dat Extreme gradient boosting de beste resultaten gaf, met een accuracy van 78\%, gevolgd door Adaptive Boosting (77\%) en Random Forest (75\%). Onderzoek van \autocite{Khan2017} gebruikt dan weer een Ensemble systeem. 

\subsubsection{Extreme Gradient Boost}
\label{subsub:xgboost}
Extreme Gradient Boost, ofwel XGBoost, is een supervised learning model, waar trainingdata, met meerdere features, x\textsubscript{i} wordt gebruikt om target waarde y\textsubscript{i} te voorspellen \autocite{XGBoost2023}. Uit onderzoek van \autocite{Chen2023}, waarbij vermoeidheid werd voorspeld op basis van gezichtsafbeeldingen met XGBoost, wordt een boom aangemaakt op basis van de geleverde features. Dit model gaf een hoge accuracy en is minder gevoelig voor noise. 

\subsubsection{Adaptive Boosting}
\label{subsub:adaboost}
De Adaptive Boost, of AdaBoost, is typisch een classificatie tussen twee klassen. Om het herkenningsprobleem met meerdere klassen op te lossen, kan een majority voting (MV) strategie worden gebruikt om alle paarsgewijze classificatieresultaten te combineren. Hierbij kiezen we de meest voorkomende klasse als voorspelling. AdaBoost is een adaptief algoritme om een reeks classificeerders te boosten, in die zin dat de gewichten dynamisch worden bijgewerkt op basis van de fouten in eerdere leerresultaten \autocite{Guo2001}.

\subsubsection{Random Forest Classifier}
\label{subsub:randomforest}
Random forests zijn verzamelingen decision trees die elk getraind zijn op een willekeurig gekozen deelverzameling van de beschikbare data. Deze decision trees worden willekeurig door de trainingsvoorbeelden die aan elke boom worden gegeven,  maar ook door een willekeurige subset van tests die beschikbaar zijn voor optimalisatie op elke node \autocite{Fanelli2012}. 
Een voorbeeld voor het gebruik van Random Forest in gezichtsanalyse is figuur {~\ref{fig:randomforest}}. Hierbij wordt een boom getoond voor het  schatten van de houding van het hoofd. 
\begin{figure}
    \centering
    \includegraphics[width=\columnwidth]{graphics/headposition.png}
    \caption{\label{fig:randomforest}Voorbeeld van een random forest voor het schatten van de houding van het hoofd\autocite{Fanelli2012}.}
\end{figure}

\subsubsection{Ensemble}
\label{subsub:ensemble}
Een set van 30 Lineaire Discriminant Analyse (LDA) dienen als basis classifiers en dragen bij tot het ensemble in het model van \textcite{Khan2017}. Het ensemble gebruikt de gecombineerde classificatiecapaciteiten van de base classificiers, zo verbeteren we de algehele prestaties. De volledige pipeline voor de gezichtsanalyse vinden we in figuur {~\ref{fig:ensemble}}. 
\begin{figure}
    \centering
    \includegraphics[width=\columnwidth]{graphics/ensemble.png}
    \caption{\label{fig:ensemble}Robuust Gezichtsherkenning systeem\autocite{Khan2017}.}
\end{figure}

\subsection{Bestaande uitdagingen}
Ondanks het onderzoekwerk, heersen er nog vele uitdagingen op het vlak van gezichtsanalyse. Deze uitdagingen zijn het gevolg van verschillende factoren zoals gezichtsuitdrukkingen, ruis, belichting, et cetera. Om de nauwkeurigheid van de gezichtsherkenning te verbeteren, is het belangrijk om taken van de gezichtsanalyse met elkaar te correleren. Het is bijvoorbeeld zeer waarschijnlijk dat mannen een baard of snor kunnen hebben, maar vrouwen en kinderen niet \autocite{Siddiqi2022}. Ook is het detecteren van de locatie waar de gezichten zich bevinden op de afbeelding een uitdaging. Dit wordt vaak meegenomen in de preprocessing stap van het analysesysteem \autocite{Jiang2008}.
De problemen situeren zich niet enkel op het vlak van de afbeelding, maar ook bij de machine learning modellen. Eén van de grootste problemen hierbij is overfitting. Dit gebeurt wanneer we het algoritme te veel trainen, waardoor het te hard lijkt op de trainingdata. Hierdoor zal het systeem falen wanneer we nieuwe data willen classificeren \autocite{Coppens2018}.

% Voor literatuurverwijzingen zijn er twee belangrijke commando's:
% \autocite{KEY} => (Auteur, jaartal) Gebruik dit als de naam van de auteur
%   geen onderdeel is van de zin.
% \textcite{KEY} => Auteur (jaartal)  Gebruik dit als de auteursnaam wel een
%   functie heeft in de zin (bv. ``Uit onderzoek door Doll & Hill (1954) bleek
%   ...'')


%---------- Methodologie ------------------------------------------------------
\section{Methodologie}%
\label{sec:methodologie}
Figuur {~\ref{fig:methodologie}} geeft een visuele representatie van de opgestelde methodologie.
\begin{figure}
    \centering
    \includegraphics[width=\columnwidth]{graphics/flowchart.PNG}
    \caption{\label{fig:methodologie}Methdologie voor het onderzoek.}
\end{figure}

\subsection{Requirements}
\label{sub:requirements}
In de eerste week wordt nagevraagd aan belanghebbenden van IntelliProve aan welke criteria de modellen moeten voldoen. Alle data (gezichtsfoto's) wordt verzameld. Er wordt onder andere nagegaan over welke functionaliteiten de modellen moeten beschikken en wat de verwachte prestatievereisten zijn. 
Als resultaat verwerven we een lijst van alle functionele en niet-functionele requirements, geordend volgens belang. 

\subsection{Literatuurstudie}
\label{sub:literatuurstudie}
De literatuurstudie omvat een diepgaande verkenning van facial analysis technieken en machine learning modellen. 
Deze fase biedt inzicht in de verschillende methoden voor het extraheren van gezichtskenmerken en image preprocessing technieken, specifiek met betrekking tot het schatten van leeftijd en geslacht.
Het doel is om kennis uit bestaand onderzoek te vergaren om effectieve methodologieën te identificeren in de huidige benaderingen van facial analysis. 
Het eindresultaat van deze fase, die 3 weken duurt, is een samenvatting van de belangrijkste bevindingen uit de literatuurstudie, die als basis zal dienen voor de proof-of-concept.
\subsection{Proof-of-concept}
\label{sub:proof-of-concept}
Deze fase start met het verzamelen en analyseren van de datasets. Technieken zoals normalisatie en scaling van de features, feature-extractie en data augmentation worden gebruikt om de dataset voor te bereiden op modeltraining.
Vervolgens worden verschillende machine learning algoritmen geselecteerd op basis van de bevindingen uit de literatuurstudie. De modellen worden getraind en geëvalueerd op basis van de vooropgestelde requirements. 
De pipeline wordt beoordeeld op betrouwbaarheid en precisie op basis van de opgestelde requirements. Metrieken zoals accuracy, precision, recall, F1-score en ROC curves worden gebruikt om de modellen te evalueren.
Er worden cross-validatie technieken gebruikt om de robuustheid van de modellen en de generalisatie van nieuwe data te testen. 
Deze fase vereist dan ook veel tijd en zal 6 weken duren. Het resultaat van deze fase is een proof-of-concept die bestaat uit een machine learning pipeline, die stappen voor het preprocessen van gegevens en getrainde modellen integreert voor het schatten van leeftijd en geslacht op basis van gezichtsfoto's.

\subsection{Conclusie}
\label{sub:conclusie}
In de conclusiefase worden de resultaten van de evaluatie grondig geanalyseerd. De prestaties van de ontwikkelde modellen worden beoordeeld, waarbij hun sterke punten en beperkingen worden benadrukt. Het belang van een nauwkeurige schatting van leeftijd en geslacht bij de beoordeling van de geestelijke gezondheid en de implicaties voor de gezondheidszorg worden besproken. Er worden aanbevelingen gedaan voor mogelijke verbeteringen of toekomstige onderzoeksrichtingen op basis van de bevindingen en beperkingen van het project.
\subsection{Afwerken scriptie}
\label{sub:afwerken_scriptie}
De laatste fase, die 2 weken duurt, omvat het afwerken van de bachelorproef. Dit is het eindresultaat van het geleverde onderzoek met een proof-of-concept die zal worden ingediend. 


%---------- Verwachte resultaten ----------------------------------------------
\section{Verwacht resultaat, conclusie}%
\label{sec:verwachte_resultaten}

Als resultaat van het onderzoek wordt een proof-of-concept opgesteld die een robuuste machine learning pipeline implementeert. We streven naar een accuracy van 75\%, wat al haalbaar bleek uit onderzoek van \textcite{Sanil2023}. Er wordt verwacht dat de modellen van Boosting, Random Forest en een ensemble de hoogste scores en accuracy opleveren. We geven naast de proof-of-concept ook een prestatie-evaluatie van de gebruikte modellen met uitgebreide analyse en vergelijking. De pipeline kan worden beoordeeld op betrouwbaarheid en precisie op basis van de opgestelde requirements. Metrieken zoals accuracy, precision, recall, F1-score en ROC-curves worden gebruikt om de modellen te evalueren. Onderstaand een mock-up van de scores met op figuur {~\ref{fig:ROC}} een ROC-curve. Dit onderzoek is van bijzonder belang voor het verbeteren van de beoordeling van de geestelijke gezondheidszorg door middel van camera-gebaseerde gezondheidsmetingen en levert een bijdrage aan de vooruitgang op dit gebied door gebruik te maken van geavanceerde algoritmen om leeftijd en geslacht nauwkeurig te voorspellen aan de hand van gezichtsopnames. Het is een opstap naar verder onderzoek in gezichtsanalyse. 

\begin{center}
    \begin{tabular}{||c c c c c||} 
        \hline
        Model & Accuracy & Precision & Recall & F1-Score \\ 
        \hline
        Model A & 0.72 & 0.50 & 0.80 & 0.82 \\ 
        \hline
        Model B & 0.52 & 0.25 & 0.90 & 0.39 \\
        \hline
        Model C & 0.61 & 0.70 & 0.40 & 0.51 \\
    \end{tabular}
\end{center}

\begin{figure}
    \centering
    \includegraphics[width=\columnwidth]{graphics/ROCcurve.png}
    \caption{\label{fig:ROC}ROC curve\autocite{Gomede2023}.}
\end{figure}

    
    %%---------- Andere bijlagen --------------------------------------------------
    % TODO: Voeg hier eventuele andere bijlagen toe. Bv. als je deze BP voor de
    % tweede keer indient, een overzicht van de verbeteringen t.o.v. het origineel.
    %\input{...}
    
    %%---------- Backmatter, referentielijst ---------------------------------------
    
    
    \backmatter{}
    
    \setlength\bibitemsep{2pt} %% Add Some space between the bibliograpy entries
    \printbibliography[heading=bibintoc]
    
\end{document}
