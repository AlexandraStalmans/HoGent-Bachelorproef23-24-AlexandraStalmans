%==============================================================================
% Sjabloon onderzoeksvoorstel bachproef
%==============================================================================
% Gebaseerd op document class `hogent-article'
% zie <https://github.com/HoGentTIN/latex-hogent-article>

% Voor een voorstel in het Engels: voeg de documentclass-optie [english] toe.
% Let op: kan enkel na toestemming van de bachelorproefcoördinator!
\documentclass{hogent-article}

% Invoegen bibliografiebestand
\addbibresource{voorstel_references.bib}

% Informatie over de opleiding, het vak en soort opdracht
\studyprogramme{Professionele bachelor toegepaste informatica}
\course{Bachelorproef}
\assignmenttype{Onderzoeksvoorstel}

\academicyear{2023-2024}

% TODO: Werktitel
\title{Facial Image Analysis voor het schatten van leeftijd en geslacht: Machine-Learning Approach}

% TODO: Studentnaam en emailadres invullen
\author{Alexandra Stalmans}
\email{alexandra.stalmans@student.hogent.be}


% TODO: Geef de co-promotor op
\supervisor[Co-promotor]{T.Sanglet (IntelliProve, \href{mailto:tanguy.sanglet@intelliprove.com}{tanguy.sanglet@intelliprove.com})}

% Binnen welke specialisatierichting uit 3TI situeert dit onderzoek zich?
% Kies uit deze lijst:
%
% - Mobile \& Enterprise development
% - AI \& Data Engineering
% - Functional \& Business Analysis
% - System \& Network Administrator
% - Mainframe Expert
% - Als het onderzoek niet past binnen een van deze domeinen specifieer je deze
%   zelf
%
\specialisation{AI \& Data Engineering}
\keywords{Facial Image Analysis, Machine Learning, Facial Recognition, Predictive Modelling}

\begin{document}

\begin{abstract}
  Hier schrijf je de samenvatting van je voorstel, als een doorlopende tekst van één paragraaf. Let op: dit is geen inleiding, maar een samenvattende tekst van heel je voorstel met inleiding (voorstelling, kaderen thema), probleemstelling en centrale onderzoeksvraag, onderzoeksdoelstelling (wat zie je als het concrete resultaat van je bachelorproef?), voorgestelde methodologie, verwachte resultaten en meerwaarde van dit onderzoek (wat heeft de doelgroep aan het resultaat?).
  Gezichtsanalyse heeft de laatste jaren veel aandacht gekregen vanwege de brede toepassingen op verschillende gebieden, zoals gezondheidszorg, beveiliging en marketing.
\end{abstract}

\tableofcontents

% De hoofdtekst van het voorstel zit in een apart bestand, zodat het makkelijk
% kan opgenomen worden in de bijlagen van de bachelorproef zelf.
%---------- Inleiding ---------------------------------------------------------

\section{Introductie}%
\label{sec:introductie}

De onderzoeksvraag werd aangeboden door het bedrijf IntelliProve. IntelliProve biedt online gezondheidsoplossingen, een software die in staat is om binnen enkele seconden nauwkeurig gezondheidsparameters te bepalen, gebaseerd op een optische meting van het gezicht.  
Het doel van de bachelorproef is het ontwikkelen en implementeren van een robuust systeem voor het schatten van de leeftijd en het geslacht van personen op basis van gezichtsfoto's, met behulp van machine learning-technieken. 
Dit project is van bijzonder belang voor het verbeteren van de beoordeling van de geestelijke gezondheidszorg door middel van camera-gebaseerde gezondheidsmetingen. Het onderzoek beoogt bij te dragen aan de vooruitgang op dit gebied door gebruik te maken van geavanceerde algoritmen om leeftijd en geslacht nauwkeurig te voorspellen aan de hand van gezichtsbeelden.  
De literatuurstudie biedt een inzicht in facial analysis, de bestaande machine learning modellen en hun functionaliteiten. De proof-of-concept zal bestaan uit het ontwikkelen van een machine learning pipeline dat in staat is om leeftijd en geslacht te voorspellen op basis van bestaande datasets. De pipeline omvat verschillende image preprocessing technieken om de dataset voor te bereiden op de modeltraining. Om betrouwbaarheid en accuracy te garanderen, worden de modellen verfijnd en geoptimaliseerd om de hoogst mogelijke nauwkeurigheid te bereiken bij het schatten van leeftijd en geslacht.

\section{State-of-the-art}%
\label{sec:state-of-the-art}

Hier beschrijf je de \emph{state-of-the-art} rondom je gekozen onderzoeksdomein, d.w.z.\ een inleidende, doorlopende tekst over het onderzoeksdomein van je bachelorproef. Je steunt daarbij heel sterk op de professionele \emph{vakliteratuur}, en niet zozeer op populariserende teksten voor een breed publiek. Wat is de huidige stand van zaken in dit domein, en wat zijn nog eventuele open vragen (die misschien de aanleiding waren tot je onderzoeksvraag!)?

Je mag de titel van deze sectie ook aanpassen (literatuurstudie, stand van zaken, enz.). Zijn er al gelijkaardige onderzoeken gevoerd? Wat concluderen ze? Wat is het verschil met jouw onderzoek?

Verwijs bij elke introductie van een term of bewering over het domein naar de vakliteratuur, bijvoorbeeld~\autocite{Hykes2013}! Denk zeker goed na welke werken je refereert en waarom.

Draag zorg voor correcte literatuurverwijzingen! Een bronvermelding hoort thuis \emph{binnen} de zin waar je je op die bron baseert, dus niet er buiten! Maak meteen een verwijzing als je gebruik maakt van een bron. Doe dit dus \emph{niet} aan het einde van een lange paragraaf. Baseer nooit teveel aansluitende tekst op eenzelfde bron.

Als je informatie over bronnen verzamelt in JabRef, zorg er dan voor dat alle nodige info aanwezig is om de bron terug te vinden (zoals uitvoerig besproken in de lessen Research Methods).

% Voor literatuurverwijzingen zijn er twee belangrijke commando's:
% \autocite{KEY} => (Auteur, jaartal) Gebruik dit als de naam van de auteur
%   geen onderdeel is van de zin.
% \textcite{KEY} => Auteur (jaartal)  Gebruik dit als de auteursnaam wel een
%   functie heeft in de zin (bv. ``Uit onderzoek door Doll & Hill (1954) bleek
%   ...'')

Je mag deze sectie nog verder onderverdelen in subsecties als dit de structuur van de tekst kan verduidelijken.

%---------- Methodologie ------------------------------------------------------
\section{Methodologie}%
\label{sec:methodologie}

\subsection{Requirements}
\label{sub:requirements}
In de eerste week wordt nagevraagd aan belanghebbenden van IntelliProve aan welke criteria de modellen moeten voldoen. Alle data (gezichtsfoto's) worden verzameld. Er wordt onder andere nagegaan over welke functionaliteiten de modellen moeten beschikken en wat de verwachte prestatievereisten zijn. 
Als resultaat verwerven we een lijst van alle functionele en niet-functionele requirements, geordend volgens belang. 

\subsection{Literatuurstudie}
\label{sub:literatuurstudie}
De literatuurstudie omvat een diepgaande verkenning van facial analysis technieken en machine learning modellen. 
Deze fase biedt inzicht in de verschillende methoden voor het extraheren van gezichtskenmerken en image preprocessing technieken, specifiek met betrekking tot het schatten van leeftijd en geslacht.
Het doel is om kennis uit bestaand onderzoek te vergaren om effectieve methodologieën te identificeren in de huidige benaderingen van facial analysis. 
Het eindresultaat van deze fase, die 3 weken duurt, is een samenvatting van de belangrijkste bevindingen uit de literatuurstudie, die als basis zal dienen voor de proof-of-concept.
\subsection{Proof-of-concept}
\label{sub:proof-of-concept}
Deze fase start met het verzamelen en analyseren van de datasets. Technieken zoals normalisatie, scaling, feature-extractie en data augmentation worden gebruikt om de dataset voor te bereiden op modeltraining.
Vervolgens worden verschillende machine learning algoritmen geselecteerd op basis van de bevindingen uit de literatuurstudie. De modellen worden getraind en geëvalueerd op basis van de vooropgestelde requirements. 
De pipeline wordt beoordeeld op betrouwbaarheid en precisie op basis van de opgestelde requirements. Metrieken zoals accuracy, precision, recall, F1-score en ROC-curves worden gebruikt om de modellen te evalueren.
Er worden cross-validatie technieken gebruikt om de robuustheid van de modellen en de generalisatie van nieuwe data te testen. 
Deze fase vereist dan ook veel tijd en zal 6 weken duren. Het resultaat van deze fase is een proof-of-concept die bestaat uit een machine learning pipeline, die stappen voor het preprocessen van gegevens en getrainde modellen integreert voor het schatten van leeftijd en geslacht op basis van gezichtsfoto's.

\subsection{conclusie}
\label{sub:conclusie}
In de conclusiefase worden de resultaten van de evaluatie grondig geanalyseerd. De prestaties van de ontwikkelde modellen worden beoordeeld, waarbij hun sterke punten en beperkingen worden benadrukt. Het belang van een nauwkeurige schatting van leeftijd en geslacht bij de beoordeling van de geestelijke gezondheid en de implicaties voor de gezondheidszorg worden besproken. Er worden aanbevelingen gedaan voor mogelijke verbeteringen of toekomstige onderzoeksrichtingen op basis van de bevindingen en beperkingen van het project.
\subsection{Afwerken scriptie}
\label{sub:afwerken_scriptie}
De laatste fase, die 2 weken duurt, omvat het afwerken van de bachelorproef. Dit is het eindresultaat van het geleverde onderzoek met een proof-of-concept die zal worden ingediend. 


%---------- Verwachte resultaten ----------------------------------------------
\section{Verwacht resultaat, conclusie}%
\label{sec:verwachte_resultaten}

Hier beschrijf je welke resultaten je verwacht. Als je metingen en simulaties uitvoert, kan je hier al mock-ups maken van de grafieken samen met de verwachte conclusies. Benoem zeker al je assen en de onderdelen van de grafiek die je gaat gebruiken. Dit zorgt ervoor dat je concreet weet welk soort data je moet verzamelen en hoe je die moet meten.

Wat heeft de doelgroep van je onderzoek aan het resultaat? Op welke manier zorgt jouw bachelorproef voor een meerwaarde?

Hier beschrijf je wat je verwacht uit je onderzoek, met de motivatie waarom. Het is \textbf{niet} erg indien uit je onderzoek andere resultaten en conclusies vloeien dan dat je hier beschrijft: het is dan juist interessant om te onderzoeken waarom jouw hypothesen niet overeenkomen met de resultaten.


\printbibliography[heading=bibintoc]
\end{document}